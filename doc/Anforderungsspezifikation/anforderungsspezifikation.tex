\documentclass[a4paper]{article}
\usepackage[margin=3cm]{geometry}
\usepackage[utf8]{inputenc}
\usepackage{cmbright}
\usepackage{hyperref}
\usepackage{booktabs}
\usepackage[ngerman]{babel}
\usepackage{parskip}
\usepackage{graphicx}
\usepackage{minted}
\usepackage{pdflscape}
\usepackage{array}
\usepackage{tabulary}
\usepackage{multicol}
\usepackage{pgfgantt}
\usepackage{pgf-umlcd}
\usepackage{enumitem}

% Page breaks between sections
\let\oldsection\section
\renewcommand\section{\clearpage\oldsection}

% JIRA/Confluence shortcuts
\def\jiraurl{https://jira.keltec.ch/jira}
\def\confluenceurl{https://jira.keltec.ch/wiki}
\newcommand{\jiraissue}[1]{\href{\jiraurl/projects/EPJ/issues/EPJ-#1}{EPJ-#1}}
\newcommand{\fulljiraissue}[1]{EPJ-#1 (\url{\jiraurl/projects/EPJ/issues/EPJ-#1})}

% Tools
\newcommand{\tool}[2]{\emph{#1\footnote{\url{#2}}}}

\begin{document}
	\title{
		Projekt: kitovu \\
		\Large{Anforderungsspezifikation} \\[3em]
		\includegraphics[width=20em]{../../img/logo/kitovu.jpg}
	}
	\author{
		Florian Bruhin \\ \url{florian.bruhin@hsr.ch} \and
		Méline Sieber \\ \url{meline.sieber@hsr.ch} \and
		Nicolas Ganz \\ \url{nicolas.ganz@hsr.ch} 
		}
	\date{\today}
	
	\maketitle

\section*{Änderungsgeschichte}

\begin{tabulary}{\linewidth}{llLl}
	\toprule
	Datum & Version & Änderung & AutorIn \\
	\midrule
	04.03.2018 & 1.0 & Dokument erstellt, Grundgerüst von Template übernommen, funktionale Anforderungen verfasst & Méline Sieber \\
	06.03.2018 & 1.1 & Überarbeitung & alle \\
	08.03.2018 & 1.2 & Lektorat & Méline Sieber\\
	09.03.2018 & 1.3 & Abgabe & Florian Bruhin \\
	\bottomrule
\end{tabulary}
\pagebreak

\section{Einführung}
Dieses Dokument beschreibt Umfang und Funktionalitäten von \emph{kitovu}. Danach veranschaulicht diese Anforderungsspezifikation, wer \emph{kitovu} verwendet. Das erfolgt anhand von ``Use Cases'', einerseits in einem kurzen Format (\emph{brief}), andererseits in einer ausführlichen Beschreibung (\emph{fully dressed}). Eine Domainanalyse zeigt wesentliche konzeptionelle Klassen und ihre Zusammenhänge auf. Weitere Anforderungen detaillieren, welche Qualitätsmerkmale und Schnittstellen verwendet werden sowie geltende Randbedingungen.

\subsection{Gültigkeitsbereich}
Die vorliegende Anforderungsspezifikation ist für das Engineering-Projekt im Frühlingssemester 2018 gültig. Falls dem Projekt grössere Veränderungen widerfahren, wird das Dokument dementsprechend angepasst. Umfassende Änderungen werden am Anfang des Dokuments protokolliert.

\subsection{Referenzen}

% <Liste aller verwendeten und referenzierten Dokumente, Bücher, Links, usw.>
% <Referenz auf ein Glossar Dokument, wo alle Abkürzungen und unklaren Begriffe erklärt werden>
% <Die Quellen / Referenzen sollten mit dem Word Tool automatisch erstellt werden>

Die Anforderungsspezifikation ist eng mit der Domainanalyse und anderen Dokumenten verbunden. Die folgende Tabelle listet die wichtigsten Referenzen auf.

\begin{tabulary}{\linewidth}{ll}
	Confluence & \url{\confluenceurl} \\
	Domainanalyse & FIXME \\
	Draw.io & \url{https://www.draw.io/} \\
	Github-Repository von \emph{kitovu} & \url{https://github.com/kitovu-bot/kitovu} \\
	JIRA	& \url{\jiraurl} \\
	Moodle & \url{https://moodle.hsr.ch} \\
	OpenHSR Connect & \url{https://github.com/openhsr/connect} \\
	Studentenportal & \url{https://studentenportal.ch/} \\
	Switch AAI (Authentication and Authorization Infrastructure)& \url{https://www.switch.ch/aai/} \\
	
\end{tabulary}

Beim Logo auf der Titelseite handelt es sich um eine stark überarbeitete Version eines GIFs (\url{https://www.animateit.net/details.php?image_id=8990}). Urheber und Copyright sind nicht auffindbar.

\pagebreak
\section{Allgemeine Beschreibung}

\subsection{Produktperspektive}
%<Produkt Perspektive beschreiben>

\subsection{Produktfunktion}
%<Allgemeine Beschreibung der Funktionen>
\emph{Kitovu} ist ein Client, der von verschiedenen Plattformen ausgewählte HSR-Unterrichtsmaterialien auf den eigenen Rechner synchronisiert. Er läuft auf allen gängigen Betriebssystemen und funktioniert nicht nur für den HSR-Skripteserver, sondern ist auch erweiterbar für verschiedene Plattformen.

Unser Projekt bindet primär den Skripteserver ein. Der Kommandozeilen-basierte Client funktioniert mittels Profilen zu unterschiedlichen Plattformen (Moodle, Skripteserver, Studentenportal). Pro Profil sind Verbindungsdaten und eventuelle Login-Credentials im Client hinterlegt, alle Profile sind in einer Konfigurationsdatei gespeichert. Die Datei-Synchronisation erfolgt immer nur von Server zu Client, erfolgreiche und misslungene Datentransfers werden protokolliert. Ein rudimentäres GUI dient als Proof-of-Concept.

Pro Profil lässt sich Folgendes definieren:

\begin{itemize}
	\item welche Ordner/Dateien synchronisiert werden.
	\item welche Ordner/Dateien von der Synchronisation ausgeschlossen werden.
	\item wie mit Duplikaten/lokal bestehenden Dateien umgegangen wird.
\end{itemize}

\emph{Kitovu} ist ausbaubar und damit modular: Zusätzlich zu den beiden Plattformen (Skripteserver; Moodle oder Studentenportal) können in zukünftigen Projekten beliebig viele Plattformen als separates Plugin bzw. Profil realisiert werden.

Optionale Features:

\begin{itemize}
	\item Moodle und/oder das Studentenportal.\footnote{Die Implementation von Moodle oder des Studentenportals ist abhängig von verschiedenen Risiken, die der Projektplan genauer ausführt.} 
	\item Komplettes GUI, das der Funktionalität des Kommandozeilen-Clients entspricht.
\end{itemize}

\subsection{Benutzercharakteristik}
%<Zielgruppe des Produktes>
Studentinnen und Studenten verwenden \emph{kitovu}, um ihre Unterrichtsmaterialien auf ihren Rechnern à jour zu halten. Sie verwenden verschiedene Betriebssysteme (Windows, macOS, Linux). Ihre Erfahrung mit der Kommandozeile ist unterschiedlich; manche verwenden sie nie, andere benutzen sie zur Standardinteraktion mit ihrem Betriebssystem.

Dozentinnen und Dozenten können ebenfalls \emph{kitovu} verwenden, sie sind jedoch nicht die primäre Zielgruppe.

\subsection{Einschränkungen}
%<Wo sind die Grenzen des Produkts>
Da die Projekt-Zeitspanne kurz ist, ist die Kernfunktion von \emph{kitovu} ein Client, der über die Kommandozeile bedient wird. Es ist vorerst nur eine rudimentäre grafische Benutzeroberfläche geplant. Falls jedoch genügend Zeit bleibt, baut das Team die Benutzeroberfläche aus. Uns ist bewusst, dass wir damit einen Teil der Studierenden ausschliessen, nicht alle können mit der Kommandozeile umgehen. Aufgrund der Zeitbeschränkung müssen wir das in Kauf nehmen, sehen es aber als erste Priorität. Falls genügend Zeit bleiben sollte, erweitern wir den grafischen Client auf die Funktionalität des Terminal-Clients.

Weitere Einschränkungen sind das Studentenportal und Moodle. Die Gründe dazu beschreibt bereits der Projektplan ausführlich.


\subsection{Annahmen}
%<Was ist unklar und wird angenommen bezüglich des Projektes oder des Produktes>
Eine wichtige Voraussetzung sind bereits bestehende Accounts. Wir gehen davon aus, dass die Studierenden bereits einen HSR-Account besitzen und sich damit sowohl per VPN von zu Hause als auch direkt an der HSR mit dem Skripteserver verbinden können. Das Studentenportal\footnote{\url{https://studentenportal.ch/}} verlangt einen separaten Account, der Zugriff auf Moodle\footnote{\url{https://moodle.hsr.ch}} erfolgt via Switch-AAI, der Authentisierung- und Autorisierungsschnittstelle für alle Schweizer Hochschulen\footnote{\url{https://www.switch.ch/aai/}}.

Wir müssen annehmen, dass ein Teil der Studentinnen und Studenten damit vertraut ist, die Kommandozeile zu bedienen. Die Konfiguration von \emph{kitovu} erfolgt über ein einzelnes Konfigurationsfile -- wir müssen ebenfalls davon ausgehen, dass die Studierenden damit umgehen können. Folglich schliesst ein Kommandozeilen-basierter Client einen Teil der Studierenden aus.

\subsection{Abhängigkeiten}
%<Von welchen Faktoren hängt das Produkt ab>
\emph{Kitovu} steht und fällt mit der Anbindung an die Plattformen, also den HSR-Skripteserver sowie die optionalen Plattformen Moodle oder Studentenportal.


\pagebreak
\begin{landscape}
  \thispagestyle{empty}
  \section{Domainanalyse}
  
  FIXME: schreib was hin
  
  \subsection{Domainmodell}
  
  % https://github.com/xuyuan/pgf-umlcd/blob/master/pgf-umlcd-manual.pdf
  \begin{tikzpicture}
    \begin{class}{File}{0,0}
      \attribute{path}
      \attribute{localDigest}
      \attribute{remoteDigest}
      \attribute{excluded : bool}
    \end{class}
  
    \begin{class}{AbstractSyncPlugin}{9,-0.4}
      \attribute{name : string}
      \attribute{version}
    \end{class}

    \association{File}{}{0..*}{AbstractSyncPlugin}{0..1}{}
  
    \begin{class}{MoodleSyncPlugin}{9, -5}
      \inherit{AbstractSyncPlugin}
    \end{class}
  
    \begin{class}{StudentenportalSyncPlugin}{6, -3}
      \inherit{AbstractSyncPlugin}
    \end{class}
  
    \begin{class}{SkripteserverSyncPlugin}{12, -3}
      \inherit{AbstractSyncPlugin}
    \end{class}

    \begin{class}{Profile}{14,2}
    \end{class}

    \begin{class}{Configuration}{19,0}
    \end{class}

    \association{Profile}{}{1}{Configuration}{1}{}
    \association{Profile}{}{1}{AbstractSyncPlugin}{1}{}

  \end{tikzpicture}
\end{landscape}

\pagebreak
\section{Funktionale Anforderungen}

\subsection{Aktoren und Stakeholder}
%<Aufzählung und Beschreibung der Aktoren & Stakeholder>

\begin{tabulary}{\linewidth}{lL}
	\toprule
	Aktor & Beschreibung\\
	\midrule
	Studentin & HSR-Studentin, die über einen HSR-Account verfügt.\\
	Skripteserver & HSR-Plattform, in der Dozenten Unterrichtsmaterialien zur Verfügung stellen.\\
	Studentenportal & Portal für eigene Materialien der HSR-Studierenden.\\
	Switch-AAI & Supporting Actor; ermöglicht Zugriff auf Moodle \\	
	Moodle & Zugriff nur via Switch-AAI-Token\\
	
	\bottomrule
\end{tabulary}

\subsection{Use-Case-Diagramm}
%<Use Case Diagramm>

FIXME insert definitive use case diagram, schön anordnen

\includegraphics[width=40em]{./uc_diagram_kitovu.png}

\subsection{Beschreibungen (Brief)}
%<Alle Use Cases in einzelnen Kapiteln beschreiben im Brief Format>
\begin{description}
	
\item[Use Case 1: Konfiguration editieren:] Die Studentin legt in einer Konfigurationsdatei alle Plattformen und die entsprechenden Verbindungsdaten fest, mit der auf die externen Plattformen zugegriffen wird.

\item[Use Case 2: Dateien auswählen:] Die Studentin legt in der Konfigurationsdatei fest, welche Ordner und Dateien synchronisiert werden sollen.

\item[Use Case 3: Verbindung erstellen:] Die Studentin stellt eine Verbindung zur externen Plattform her und authentisiert sich dort.

\item[Use Case 4: Synchronisation:] Die Studentin synchronisiert die Dateien von der externen Plattform auf ihren Rechner.
\end{description}

\pagebreak
\subsection{Beschreibungen (Fully Dressed)}
%<Spezielle und wichtige Use Cases in einzelnen Kapiteln beschreiben im Fully Dressed Format>

FIXME Méline: Use Cases umschreiben, Brief-Version folgend. Überall Studentin

\SetEnumitemKey{uclist}{labelwidth=13em,leftmargin=13.5em}

\subsubsection{UC1: Konfiguration editieren}
\begin{description}[uclist]
  \item[Goal] Der Student editiert die Konfiguration, in der alle externen Plattform enthalten ist.
  %\item[Level] FIXME
  \item[Primary Actor] Student
  \item[Trigger] Der Student möchte aktuelle Unterrichtsmaterialien auf seinem Rechner haben.
  \item[Stakeholders and Interests]
    \begin{description}
      \item[Der Student] legt eine neue Plattform an.
      \item[Das System] informiert den Studenten, welche Plattformen unterstützt werden und stellt sicher, dass die Plattform bereits implementiert ist.
    \end{description}
  \item[Preconditions] Der Student besitzt einen Account der externen Plattform mit entsprechenden Login-Daten.
  \item[Postconditions] Die aktualisierte Konfigurationsdatei wird gespeichert.
  \item[Main Success Scenario] Der Student hinterlegt in der Konfiguration von \emph{kitovu} zu einer externen Plattform in \emph{kitovu}. Er kann die Zugangsdaten erstellen, speichern, abändern oder das Profil ganz entfernen. Die eingegebenen Daten werden in \emph{kitovu} gespeichert.
  \item[Extensions] Die eingegebenen Daten sind ungültig. \emph{Kitovu} weist die Studentin darauf hin.
  %\item[Special Requirements] FIXME
  \item[Frequency of Occurrence] Selten, nach Bedarf.
\end{description}

\subsubsection{UC2: Dateien auswählen}
\begin{description}[uclist]
  \item[Goal] Die Studentin schreibt in die Konfigurationsdatei, welche Dateien sie auf ihren Rechner synchronisieren möchte. Sie kann ausschliessen, welche Dateien nicht synchronisiert werden.
  %item[Level] FIXME
  \item[Primary Actor] Studentin
  \item[Trigger] Die Studentin möchte nur bestimmte Dateien und Ordner auf ihrem Rechner haben.
  \item[Stakeholders and Interests]
    \begin{description}
      \item[Die Studentin] weiss, welche Dateien sie synchronisieren möchte.
      \item[Das System] kann die entsprechenden Dateien lokalisieren.
    \end{description}
  \item[Preconditions]
    \begin{itemize}[leftmargin=1em]
      \item UC1 ist erfolgt: Die Studentin besitzt einen Account der externen Plattform. Sie hat bereits ein Profil der externen Plattform in \emph{kitovu} hinterlegt.
      \item Sie weiss, welche Dateien sie synchronisieren möchte und wo sie auf der externen Plattform abgelegt sind.
    \end{itemize}
  \item[Postconditions] Die aktualisierte Konfigurationsdatei wird gespeichert.
  \item[Main Success Scenario] Die Studentin definiert für das jeweilige Profil, welche Dateien sie synchronisiert haben möchte und welche ausgeschlossen werden.
  \item[Extensions] Die angegebenen Pfade existieren nicht -- die Studentin wird in UC4 darauf hingewiesen.
  %\item[Special Requirements] asdf
  \item[Frequency of Occurrence] Zu Beginn jedes Semesters, wenn die neuen Module beginnen.
  %\item[Open Issues] asdf
\end{description}

\subsubsection{UC3: Verbindung erstellen}
\begin{description}[uclist]
  \item[Goal] Um auf die Unterrichtsmaterialien zuzugreifen, muss der Student eine Verbindung zur externen Plattform herstellen.
  %\item[Level] FIXME
  \item[Primary Actor] Der Student
  \item[Trigger] Der Student möchte auf eine externe Plattform zugreifen.
  \item[Stakeholders and Interests]
    \begin{description}
      \item[Der Student] weiss, von welcher Plattform er welche Dateien möchte.
      \item[Das System] kann die Verbindung herstellen.
    \end{description}
  \item[Preconditions] Der Student hat in UC1 die Profildaten der externen Plattform hinterlegt.
  \item[Postconditions] Keine.
  \item[Main Success Scenario] Der Student verbindet sich zur externen Plattform, die Authentisierung ist erfolgreich.
  \item[Extensions]
    \begin{description}
      \item[Die Verbindung misslingt:] \strut \\[-1em]
        \begin{itemize}[leftmargin=1em]
          \item Nicht im HSR-Netz (HSR-Skripteserver): Die Verbindung misslingt, da sich der Student nicht im HSR-Netz befindet oder nicht per VPN verbunden ist.
          \item Moodle-Probleme: der Zugriff auf Moodle misslingt aufgrund Plattform-eigener Probleme.
          \item Die hinterlegten Login-Daten sind nicht korrekt.
        \end{itemize}
      \item[Keine Login-Daten hinterlegt:] \emph{kitovu} fordert die Studentin auf, ihre Login-Daten einzugeben.
      \item[Moodle:] Die externe Plattform verlangt, dass sich der Student pro Session authentisiert.
      \item[HSR-Skripteserver:] Die externe Plattform verlangt, dass sich der Student im HSR-Netz befindet oder bereits eine VPN-Verbindung zur Plattform aufgenommen hat.
    \end{description}
  \item[Special Requirements] Der Student verfügt über einen Account der externen Plattform.
  \item[Frequency of Occurrence] Rund jede Woche, wenn die Dozenten erneut neue Unterrichtsmaterialien bereitstellen.
  %\item[Open Issues] FIXME
\end{description}


\subsubsection{UC4: Synchronisation der Dateien}

\begin{description}[uclist]
  \item[Goal] Die Studentin synchronisiert alle benötigten Unterrichtsmaterialien auf ihren Computer.
  %\item[Level] FIXME
  \item[Primary Actor] Die Studentin
  \item[Trigger] Die Studentin möchte aktuelle Unterrichtsmaterialien auf ihrem Computer lokal abspeichern.
  \item[Stakeholders and Interests]
    \begin{description}
      \item[Die Studentin] möchte stets die Version der Dateien auf ihrem Rechner haben, die sie tatsächlich will.
      \item[Das System] stellt sicher, dass die gespeicherten Dateien konsistent und aktuell sind.
    \end{description}
  \item[Preconditions] UC3: Die Studentin hat erfolgreich eine Verbindung zur externen Plattform hergestellt.
  %\item[Postconditions] FIXME
  \item[Main Success Scenario] Die Studentin startet den Synchronisationsprozess. Danach verfügt sie über die von ihr ausgewählten Unterrichtsmaterialien auf ihrem Rechner. Eine Log-Datei protokolliert den Vorgang.
  \item[Extensions]
    \begin{description}
      \item[Dateien schon vorhanden:] Die Studentin legt in ihrem Profil fest, wie mit bereits existierenden Dateien umgegangen werden soll. Dabei kann sie wählen, ob lokal bestehende Dateien gleichen Namens umbenannt, durch eine neuere Server-Version ersetzt oder nichts synchronisiert werden soll. Sie kann auch entscheiden, bei jedem Konflikt von \emph{kitovu} gefragt zu werden, wie sie für jede Datei entscheiden möchte.
      \item[Synchronisation schlägt fehl:] Die Synchronisation wird nicht durchgeführt, die Studentin erhält eine Benachrichtigung.
    \end{description}
  %\item[Special Requirements] FIXME
  \item[Frequency of Occurrence] Rund jede Woche, wenn die Dozenten erneut neue Unterrichtsmaterialien bereitstellen.
  %\item[Open Issues] Moodle-Anbindung
\end{description}

\pagebreak
\section{Weitere Anforderungen}

\subsection{Qualitätsmerkmale}
%<Beschreibung der Qualitätsmerkmale der Software (Verweis auf ISO 9126 als Checkliste)>

\newcommand{\isourl}[2]{ISO/IEC #1\footnote{ISO/IEC #1 \url{https://www.iso.org/standard/#2.html}}}

Für die Qualitätsmerkmale haben wir uns für den Standard \isourl{9126}{16722} entschieden. Obwohl dieser bereits mit dem Standard \isourl{25010}{35733} überarbeitet wurde, verwenden wir die ältere Version, da wir damit im Modul \emph{Software-Engineering 1} bereits Erfahrung gesammelt haben.

\subsubsection{Funktionalität (functionality)}

\begin{description}
  \item[Angemessenheit (suitability)] \strut \\
    Die Funktionen von \emph{kitovu} konzentrieren sich nur auf die Synchronisation der Dateien.
    Dadurch ist die Applikation sehr spezialisiert, aber trotzdem erweiterbar.
  \item[Richtigkeit (accuracy)] \strut \\
    Es werden sämtliche Dateien, welche lokal nicht existieren und vorgängig ausgewählt wurden, synchronisiert.

    Um zu erkennen, ob sich Dateien geändert haben, sind vor von den Angaben abhängig, welche die externen Schnittstellen bieten.
    Dies bedeutet, unsere Kenntnis des Änderungszeitpunkts ist von der Genauigkeit des gewählten Datumsformat der externen Plattform abhängig. Das kann zu blinden Flecken führen: Änder etwas in derselben Sekunde, können wir das nicht als Änderung anerkennen. Zudem können wir nicht verhindern, dass Änderungszeitpunkte von Dateien nachträglich auf der externen Plattform verändert werden.
    
    Neben dem Änderungszeitpunkt gibt es auch die Möglichkeit, Hashes oder Dateigrössen zu vergleichen, um Änderungen an Dokumenten festzustellen. Je nach Plattform wählen wir diejenige Variante, die die ex grösste Genauigkeit bei den Änderungen zulässt.
  \item[Interoperabilität (interoperability)] \strut \\
    Die externen Schnittstellen werden über abgekoppelte Module angesprochen.
    Dadurch sind diese einfach testbar und austauschbar, falls eine Schnittstelle nicht mehr benötigt wird oder eine neue gewünscht ist.
  \item[Ordnungsmässigkeit (regularity)] \strut \\
    Für die Ordnungsmässigkeit halten wir uns an das Bundersgesetz über den Datenschutz (DSG\footnote{\url{https://www.admin.ch/opc/de/classified-compilation/19920153/index.html}}).
    Wir speichern die Daten immer nur auf dem Computer des Benutzers.
    Die einzigen Daten, welche an Dritte gehen, sind die Anmeldedaten für die entsprechende Schnittstelle sowie die Namen der Dateien, welche synchronisiert werden müssen.

    Sämtliche Änderungen an Daten werden von dem Benutzer ausgelöst. Durch die akkurate Benennung der Interaktionsmöglichkeiten in \emph{kitovu} wird verständlich gemacht, welche Aktion der Benutzer auslöst.

    Es wird nicht mit besonders schützenswerten Personendaten hantiert.
  \item[Sicherheit (security)] \strut \\
    Bei \emph{kitovu} werden Anmeldeinformationen (wie Passwörter und Login-Tokens) verschlüsselt auf dem System abgelegt.

    Die Ablage der synchronisierten Daten und Konfigurationen geschieht nicht verschlüsselt, sprich es liegt in der Hand des Benutzers vorsichtig damit umzugehen.
    Falls Daten, welche vertraulich sind, synchronisiert werden muss der Benutzer sicherstellen, dass er es nicht z.B. auf einen öffentlichen Share synchronisiert.
\end{description}

\subsubsection{Zuverlässigkeit (reliability)}

\begin{description}
  \item[Reife (maturity)] \strut \\
    In 99\% der Benutzer-Interaktionen sollte kein Fehler auftreten.
    Falls doch wird dieser abgefangen und eine verständliche und hilfreiche Fehlermeldung sollte dargestellt werden.

    Davon ausgenommen sind Fehler bei der Synchronisation, welche von den externen Schnittstellen ausgelöst sind.
  \item[Fehlertoleranz (fault tolerance)] \strut \\
    Falls ein Modul fehlerhaft ausgeführt wird (zum Beispiel falls ein Server nicht erreichbar ist) wird dies dem Benutzer dargestellt, aber sämtlichen anderen Module werden ausgeführt.

    Fehlerhafte Eingaben in der Konfiguration werden dem Benutzer verständlich kommuniziert.
  \item[Wiederherstellbarkeit (recoverability)] \strut \\
    Falls eine Synchronisation nur zum Teil ausgeführt werden konnte, kann die Synchronisation erneut ausgeführt werden und die nur die fehlenden Teile werden versucht zu synchronisieren.
    Dadurch muss ein lange dauernde Synchronisierung nicht erneut komplett durchgeführt werden.

    Falls sämtliche Dateien verloren gegangen sind, muss nur das Profil erneut geschrieben werden und der Rest wird automatisch synchronisiert.
\end{description}

\subsubsection{Benutzbarkeit (usability)}

\begin{description}
  \item[Verständlichkeit (understandability)] \strut \\
    Da \emph{kitovu} einen begrenzen, übersichtlichen Funktionsumfang hat, werden wir die einzelnen Optionen und Aktionen in der Benutzeroberfläche, respektive der \verb|--help|-Option des Kommandozeilenprogramm beschrieben.
  \item[Erlernbarkeit (learnability)] \strut \\
    Die einzelnen Funktionen sind übersichtlich gruppiert und beschrieben.
  \item[Bedienbarkeit (operability)] \strut \\
    Das Erstellen der Konfiguration benötigt etwas Aufwand, aber das eigentliche Synchronisieren wird mit minimalem Aufwand ermöglicht.
\end{description}

\subsubsection{Effizienz (efficiency)}

\begin{description}
  \item[Zeitverhalten (time behaviour)] \strut \\
    Das Zeitverhalten ist abhängig von den externen Schnittstellen und der Internet-Leitung.

    Falls jedoch nichts geändert hat, wird pro Modul immer nur ein Request abgesetzt, um die Änderung der Dateien bei der externen Schnittstelle herauszufinden.
    Je nach dem bietet die externe Schnittstelle eine solche Funktion nicht, wodurch wir für jede Datei ein Request absetzen müssen.
  \item[Verbrauchsverhalten (resource utilization)] \strut \\
    Die CPU-Zeit ist minim gehalten, da die meiste Zeit mit den Internet-Requests verbraucht wird.

    \emph{Kitovu} speichert die Änderungszeitpunkte oder Hashes der Dateien im RAM und auf der Festplatte ab. Damit lässt sich bei einer Synchronisation vergleichen, ob die Dateien auf der externen Plattform geändert wurden.
\end{description}

\subsubsection{Änderbarkeit (maintainability)}

\begin{description}
  \item[Analysierbarkeit (analyzability)] \strut \\
    Mit einem Debug-Modus können zusätzlich alle Informationen, die ein Modul liefert, aufgezeichnet werden.
    Dies ermöglicht eine Einschränkung des Problems auf entweder ein spezifisches Modul oder den Core-Teil von \emph{kitovu}selber.
  \item[Modifizierbarkeit (changeability)] \strut \\
    Da die ganze Applikation modular aufgebaut ist, lässt diese sich leicht um neue Module erweitern oder bestimmte entfernen.
  \item[Stabilität (stability)] \strut \\
    Eine grosse Testabdeckung (inklusive kompletter Abläufe) bieten eine grundstabilität, so dass Änderungen bei bestimmten Funktionalitäten die Tests sofort reklamieren.
  \item[Prüfbarkeit (testability)] \strut \\
    Dank dem modularen Aufbau können wir alle Module und den Core-Teil separat Testen.
    Dies ergänzen wir mit Integrations-Tests, welche den kompletten Ablauf testet.
\end{description}

\subsubsection{Übertragbarkeit (portability)}

\begin{description}
  \item[Anpassbarkeit (adaptability)] \strut \\
    Über neue Module lässt sich die Applikation einfach mit neuen Schnittstellen erweitern.
  \item[Installierbarkeit (installability)] \strut \\
    Die Applikation lässt sich über ein pip-Paket \footnote{\url{https://pypi.python.org/pypi/pip}} installieren. Für Windows (und allenfalls macOS) wird \emph{kitovu} als ausführbare Datei bereitgestellt.
    Dies installiert sämtliche Abhängigkeiten mit.
  \item[Koexistenz (co-existence)] \strut \\
    Falls eine weitere Applikation an denselben Ort Dateien schreibt, werden je nach Konfiguration die Dateien der anderen Applikation bei der nächsten Synchronisation einfach überschrieben.
    Je nach dem wie die andere Applikation die Dateien bearbeitet könnte es zu einer erneuten Synchronisation der Dateien führen, aber es funktioniert weiterhin.
  \item[Austauschbarkeit (replaceability)] \strut \\
    \emph{Kitovu} kann nicht direkt eine andere, bestehende Applikation ablösen, da wir einen Stand der Schnittstellen-Daten zwischenspeichern.
    Daher müssen die Daten zu Beginn frisch synchronisiert werden.
\end{description}

\subsection{Schnittstellen}
%<Beschreibung der Schnittstellen der Software>

Folgende Schnittstellen werden im Projekt \emph{kitovu} zu Verfügung gestellt oder verwendet:

\begin{description}
  \item[Grafische Benutzeroberfläche]
    Die grafische Benutzeroberfläche ist die Schnittstelle, mit welcher die meisten Endbenutzer mit der Applikation interagieren werden.
    Diese bietet einen einfachen Weg, um die Daten zu synchronisieren.
  \item[Kommandozeilenprogramm]
    Alternativ bieten wir ein Kommandozeilenprogramm, welches ermöglicht über einfache Befehle die Synchronisation zu starten.
    Dies ist besonders für Automatisierung oder für Endbenutzer, welche die Kommandozeile bevorzugen, gedacht.
  \item[Externe Schnittstellen]
    Wir verwenden als externe Schnittstellen das Studentenportal, das Moodle und den Skripteserver, sowie das SwitchAAI für die Autorisierung des Moodles.
\end{description}

\subsection{Randbedingungen}
%<Auflistung der wichtigsten Randbedingungen mit einer Beschreibung dazu>

Dies technischen Randbedingungen wurden für das Projekt \emph{kitovu} festgelegt:

\begin{description}
  \item[Python] Version 3.6
  \item[Betriebssystem] \emph{kitovu} unterstützt Windows und Linux, gegebenfalls macOS
\end{description}

\end{document}
