\documentclass[a4paper]{article}
\usepackage[margin=3cm]{geometry}
\usepackage[utf8]{inputenc}
\usepackage{cmbright}
\usepackage{hyperref}
\usepackage{booktabs}
\usepackage[ngerman]{babel}
\usepackage{parskip}
\usepackage{graphicx}
\usepackage{minted}
\usepackage{pdflscape}
\usepackage{array}
\usepackage{tabulary}
\usepackage{multicol}
\usepackage{pgfgantt}

% Page breaks between sections
\let\oldsection\section
\renewcommand\section{\clearpage\oldsection}

% JIRA/Confluence shortcuts
\def\jiraurl{https://jira.keltec.ch/jira}
\def\confluenceurl{https://jira.keltec.ch/wiki}
\newcommand{\jiraissue}[1]{\href{\jiraurl/projects/EPJ/issues/EPJ-#1}{EPJ-#1}}
\newcommand{\fulljiraissue}[1]{EPJ-#1 (\url{\jiraurl/projects/EPJ/issues/EPJ-#1})}

% Tools
\newcommand{\tool}[2]{\emph{#1\footnote{\url{#2}}}}

\begin{document}
	\title{
		Projekt: kitovu \\
		\Large{Anforderungsspezifikation} \\[3em]
		\includegraphics[width=20em]{../../img/logo/kitovu.jpg}
	}
	\author{
		Florian Bruhin \\ \url{florian.bruhin@hsr.ch} \and
		Méline Sieber \\ \url{meline.sieber@hsr.ch} \and
		Nicolas Ganz \\ \url{nicolas.ganz@hsr.ch} 
		}
	\date{\today}
	
	\maketitle

\section*{Änderungsgeschichte}

\begin{tabulary}{\linewidth}{llLl}
	\toprule
	Datum & Version & Änderung & AutorIn \\
	\midrule
	04.03.2018 & 1.0 & Dokument erstellt, Grundgerüst von Template übernommen & Méline Sieber \\

	\bottomrule
\end{tabulary}
\pagebreak

\section{Einführung}
Dieses Dokument beschreibt, was \emph{kitovu} genau ist. Danach veranschaulicht diese Anforderungsspezifikation, für wen der \emph{kitovu}-Client gedacht ist. Das erfolgt anhand von ``Use Cases'', einerseits in einem kurzen Format (\emph{brief}), andererseits in einer ausführlichen Beschreibung (\emph{fully dressed}). Weitere Anforderungen detaillieren, welche Qualitätsmerkmale und Schnittstellen verwendet werden sowie geltende Randbedingungen.

\subsection{Gültigkeitsbereich}
Die vorliegende Anforderungsspezifikation ist für das Engineering Projekt im Frühlingssemester 2018 gültig. Falls dem Projekt grössere Veränderungen widerfahren, wird das Dokument dementsprechend angepasst. Umfassende Änderungen werden am Anfang des Dokuments protokolliert.

\subsection{Referenzen}

% <Liste aller verwendeten und referenzierten Dokumente, Bücher, Links, usw.>
% <Referenz auf ein Glossar Dokument, wo alle Abkürzungen und unklaren Begriffe erklärt werden>
% <Die Quellen / Referenzen sollten mit dem Word Tool automatisch erstellt werden>

Die Anforderungsspezifikation ist eng verbunden mit der Domainanalyse und anderen Dokumenten. Die folgende Tabelle listet die wichtigsten Referenzen auf.

FIXME Verweise zum Projektplan einfügen, zur Domainanalyse u.a. Dokumenten, die wir mit der Anforderungsspezifikation abgeben

\begin{tabulary}{\linewidth}{ll}
	Confluence & \url{\confluenceurl} \\
	Domainanalyse & FIXME \\
	Draw.io & \url{https://www.draw.io/} \\
	Github-Repository von \emph{kitovu} & \url{https://github.com/kitovu-bot/kitovu} \\
	JIRA	& \url{\jiraurl} \\
	Moodle & \url{https://moodle.hsr.ch} \\
	OpenHSR Connect & \url{https://github.com/openhsr/connect} \\
	Studentenportal & \url{https://studentenportal.ch/} \\
	Switch AAI (Authentication and Authorization Infrastructure)& \url{https://www.switch.ch/aai/} \\
	
\end{tabulary}

Beim Logo auf der Titelseite handelt es sich um eine stark überarbeitete Version eines GIFs (\url{https://www.animateit.net/details.php?image_id=8990}). Urheber und Copyright waren nicht auffindbar.

\pagebreak
\section{Allgemeine Beschreibung}

\subsection{Produktperspektive}
%<Produkt Perspektive beschreiben>

\subsection{Produktfunktion}
%<Allgemeine Beschreibung der Funktionen>
\emph{Kitovu} ist ein Client, der von verschiedenen Plattformen ausgewählte HSR-Unterrichtsmaterialien auf den eigenen Rechner synchronisiert. Er läuft auf allen gängigen Betriebssystemen und funktioniert nicht nur für den HSR-Skripteserver, sondern ist auch erweiterbar für verschiedene Plattformen.

Unser Projekt bindet primär den Skripteserver ein. Der Terminal-basierte Client funktioniert mittels Profilen zu unterschiedlichen Plattformen (Moodle, Skripteserver, Studentenportal). Pro Profil sind Verbindungsdaten und eventuelle Login-Credentials im Client hinterlegt. Die Daten-Synchronisation erfolgt immer nur von Server zu Client, erfolgreiche und misslungene Datentransfers werden protokolliert. Ein rudimentäres GUI dient als Proof-of-Concept.

Pro Profil lässt sich Folgendes definieren:

\begin{itemize}
	\item welche Ordner/Dateien synchronisiert werden.
	\item welche Ordner/Dateien von der Synchronisation ausgeschlossen werden.
	\item wie mit Duplikaten/lokal bestehenden Dateien umgegangen wird.
\end{itemize}

\emph{Kitovu} ist ausbaubar und damit modular: Zusätzlich zu den beiden Plattformen (Skripteserver; Moodle oder Studentenportal) können in zukünftigen Projekten beliebig viele Plattformen als separates Plugin bzw. Profil realisiert werden.

Optionale Features:

\begin{itemize}
	\item Moodle und/oder das Studentenportal. Die Implementation von Moodle oder des Studentenportals ist abhängig von den weiter unten beschriebenen Risiken.
	\item Komplettes GUI, das der Funktionalität des Terminalprogramms entspricht.
\end{itemize}

\subsection{Benutzercharakteristik}
%<Zielgruppe des Produktes>
Studentinnen und Studenten verwenden \emph{kitovu}, um ihre Unterrichtsmaterialien auf ihren Rechnern à jour zu halten. Sie verwenden verschiedene Betriebssysteme (Windows, macOS, Linux). Ihre Erfahrung mit der Kommandozeile ist unterschiedlich; manche verwenden sie nie, andere benutzen sie zur Standardinteraktion mit dem Betriebssystem.

Dozentinnen und Dozenten können ebenfalls \emph{kitovu} verwenden, sie sind jedoch nicht die primäre Zielgruppe.

\subsection{Einschränkungen}
%<Wo sind die Grenzen des Produkts>
Da die Projekt-Zeitspanne kurz ist, ist die Kernfunktion von \emph{kitovu} ein Client, der über die Kommandozeile bedient wird. Es ist vorerst nur eine rudimentäre grafische Benutzeroberfläche geplant. Falls jedoch genügend Zeit bleibt, baut das Team diese aus. Uns ist bewusst, dass wir damit einen Teil der Studierenden ausschliessen, nicht alle können mit der Kommandozeile umgehen. Aufgrund der Zeitbeschränkung müssen wir das in Kauf nehmen, sehen es aber als erste Priorität, die wir implementieren, wenn bis zum Ende des Projekts genügend Zeit bleiben sollte.

Eine weitere Einschränkung sind das Studentenportal und Moodle. Die Gründe dazu beschreibt bereits der Projektplan ausführlich.


\subsection{Annahmen}
%<Was ist unklar und wird angenommen bezüglich des Projektes oder des Produktes>
Eine wichtige Voraussetzung sind bereits bestehende Accounts. Wir gehen davon aus, dass die Studierenden bereits einen HSR-Account besitzen und sich damit sowohl per VPN von zu Hause als auch direkt an der HSR mit dem Skripteserver verbinden können. Das Studentenportal\footnote{https://studentenportal.ch/} verlangt einen separaten Account, der Zugriff auf Moodle\footnote{\url{https://moodle.hsr.ch}} erfolgt via Switch-AAI, der Authentisierung- und Autorisierungsschnittstelle für alle Hochschulen\footnote{\url{https://www.switch.ch/aai/}}.

Wir müssen annehmen, dass ein Teil der Studentinnen und Studenten darin vertraut ist, die Kommandozeile zu bedienen. Die Konfiguration von \emph{kitovu} erfolgt über ein einzelnes Konfigurationsfile - wir müssen ebenfalls davon ausgehen, dass die Studierenden damit umgehen können.

\subsection{Abhängigkeiten}
%<Von welchen Faktoren hängt das Produkt ab>
\emph{Kitovu} steht und fällt mit der Anbindung an die Plattformen, also den HSR-Skripteserver sowie die optionalen Plattformen Moodle oder Studentenportal.

\pagebreak
\section{Funktionale Anforderungen}

\subsection{Aktoren und Stakeholder}
%<Aufzählung und Beschreibung der Aktoren & Stakeholder>
FIXME: wer sind die Stakeholder?

\begin{tabulary}{\linewidth}{lL}
	\toprule
	Aktor & Beschreibung\\
	Student & HSR-Student, der über einen HSR-Account verfügt.\\
	Skripteserver & HSR-Plattform, in der Dozierende Unterrichtsmaterialien zur Verfügung stellen.\\
	Studentenportal & Portal für eigene Materialien der HSR-Studierenden.\\
	Switch-AAI & Supporting Actor; ermöglicht Zugriff auf Moodle \\	
	
	\bottomrule
\end{tabulary}
\pagebreak

\subsection{Use-Case-Diagramm}
%<Use Case Diagramm>

FIXME insert definitive use case diagram, schön anordnen

\includegraphics[width=40em]{../../doc/Anforderungsspezifikation/uc_diagram_kitovu.png}

\subsection{Beschreibungen (Brief)}
%<Alle Use Cases in einzelnen Kapiteln beschreiben im Brief Format>
\begin{description}
	
\item[Use Case 1: Profil editieren (CRUD):] Die Studentin legt in einer Konfigurationsdatei die Plattform und die entsprechenden Zugangsdaten fest, mit der auf die externe Plattform zugegriffen wird.

\item[Use Case 2: Zu synchronisierende Dateien editieren (CRUD):] Die Studentin konfiguriert, welche Ordner und Dateien synchronisiert werden sollen.

\item[Use Case 3: Verbindung erstellen:] Die Studentin verifiziert sich bei der externen Plattform.

\item[Use Case 4: Synchronisation der Dateien:] Die Studentin synchronisiert die Dateien von der externen Plattform auf ihren Rechner.
\end{description}

\subsection{Beschreibungen (Fully Dressed)}
%<Spezielle und wichtige Use Cases in einzelnen Kapiteln beschreiben im Fully Dressed Format>

FIXME Tabelle in schön

\subsubsection{UC1: Profil editieren}
\begin{tabulary}{\linewidth}{lp{30em}}
	\textbf{Goal} & Der Student editiert ein Profil zu einer externen Plattform: CRUD (Create, Read Update, Delete) \\
	\textbf{Level} & FIXME \\
	\textbf{Primary Actor} & Student\\
	\textbf{Trigger} & Der Student möchte aktuelle Unterrichtsmaterialien auf seinem Rechner haben.\\
	\textbf{Stakeholders and Interests} & FIXME \\
	\textbf{Preconditions} & Der Student besitzt einen Account der externen Plattform mit entsprechenden Login-Daten.\\
	\textbf{Postconditions} & FIXME \\
	\textbf{Main Success Scenario} & Der Student erstellt ein Profil zu einer externen Plattform in \emph{kitovu}. Er kann die Zugangsdaten erstellen, speichern, abändern oder das Profil ganz entfernen. Die eingegebenen Daten werden in \emph{kitovu} gespeichert. \\
	\textbf{Extensions} & Die externe Plattform verlangt, dass sich der Student pro Session authentisiert (Moodle).\\
	\textbf{Special Requirements} & asdf \\
	\textbf{Frequency of Occurrence} & asdf \\
	\textbf{Open Issues} & Keine. \\
\end{tabulary}

\subsubsection{UC2: Dateien auswählen}
\begin{tabulary}{\linewidth}{lp{30em}}
	\textbf{Goal} & Die Studentin wählt aus, welche Dateien sie auf ihren Rechner synchronisieren möchte (CRUD)-\\
	\textbf{Level} & FIXME \\
	\textbf{Primary Actor} & Studentin\\
	\textbf{Trigger} & Die Studentin möchte nur bestimmte Dateien und Ordner auf ihrem Rechner haben.\\
	\textbf{Stakeholders and Interests} & FIXME \\
	\textbf{Preconditions} & Die Studentin besitzt einen Account der externen Plattform. Sie hat bereits ein Profil der externen Plattform in \emph{kitovu} hinterlegt.\\
	\textbf{Postconditions} & UC3 ist erfolgreich.\\
	\textbf{Main Success Scenario} & Die Studentin wählt die zu synchronisierenden Daten aus, startet den Synchronisationsprozess und verfügt danach über die von ihr ausgewählten Unterrichtsmaterialien auf ihrem Rechner.\\
	\textbf{Extensions} & Sie hat bereits eine  \\
	\textbf{Special Requirements} & asdf \\
	\textbf{Frequency of Occurrence} & Zu Beginn jedes Semesters, wenn die neuen Module beginnen. \\
	\textbf{Open Issues} & asdf \\
\end{tabulary}

\subsubsection{UC3: Verbindung erstellen}
\begin{tabulary}{\linewidth}{lp{30em}}
	\textbf{Goal} & Um auf die Unterrichtsmaterialien zuzugreifen, muss der Student eine Verbindung zu externen Plattform aufnehmen. \\
	\textbf{Level} & FIXME \\
	\textbf{Primary Actor} & Der Student \\
	\textbf{Trigger} & \url{https://github.com/kitovu-bot/kitovu} \\
	\textbf{Stakeholders and Interests} & \url{\jiraurl} \\
	\textbf{Preconditions} & Der Student hat in UC1 die Profildaten der externen Plattform hinterlegt.\\
	\textbf{Postconditions} & keine.\\
	\textbf{Main Success Scenario} & Der Student verbindet sich erfolgreich zur externen Plattform. \\
	\textbf{Extensions} & \url{https://www.switch.ch/aai/} \\
	\textbf{Special Requirements} & Der Student verfügt über einen Account der externen Plattform. \\
	\textbf{Frequency of Occurrence} & FIXME \\
	\textbf{Open Issues} & FIXME \\
\end{tabulary}


\subsubsection{UC4: Synchronisation der Dateien}

\begin{tabulary}{\linewidth}{lp{30em}}
	\textbf{Goal} & Die Studentin synchronisiert alle benötigten Unterrichtsmaterialien auf ihren Computer. \\
	\textbf{Level} & FIXME \\
	\textbf{Primary Actor} & Die Studentin \\
	\textbf{Trigger} & Die Studentin möchte aktuelle Unterrichtsmaterialien auf ihrem Computer lokal abspeichern. \\
	\textbf{Stakeholders and Interests} & FIXME \\
	\textbf{Preconditions} & Die Studentin hat erfolgreich eine Verbindung zur externen Plattform hergestellt. \\
	\textbf{Postconditions} & FIXME \\
	\textbf{Main Success Scenario} & FIXME \\
	\textbf{Extensions} & FIXME \\
	\textbf{Special Requirements} & FIXME \\
	\textbf{Frequency of Occurrence} & Rund jede Woche, wenn die Dozenten erneut neue Unterrichtsmaterialien bereitstellen.  \\
	\textbf{Open Issues} & FIXME \\
\end{tabulary}

\pagebreak
\section{Weitere Anforderungen}

\subsection{Qualitätsmerkmale}
%<Beschreibung der Qualitätsmerkmale der Software (Verweis auf ISO 9126 als Checkliste)>

Gemäss ISO 9126

\subsubsection{Funktionalität (functionality)}

\begin{description}
  \item[Angemessenheit (suitability)]
    FXME
  \item[Richtigkeit (accuracy)]
    FIXME
  \item[Interoperabilität (interoperability)]
    FIXME
  \item[Sicherheit (security)]
    FIXME
  \item[Ordnungsmässigkeit (functionality compliance)]
    FIXME
\end{description}

\subsubsection{Zuverlässigkeit (reliability)}

\begin{description}
  \item[Reife (maturity)]
    FXME
  \item[Fehlertoleranz (fault tolerance)]
    FIXME
  \item[Wiederherstellbarkeit (recoverability)]
    FIXME
  \item[Konformität (reliability compliance)]
    FIXME
\end{description}

\subsubsection{Benutzbarkeit (usability)}

\begin{description}
  \item[Verständlichkeit (understandability)]
    FIXME
  \item[Erlernbarkeit (learnability)]
    FIXME
  \item[Bedienbarkeit (operability)]
    FIXME
  \item[Attraktivität (attractiveness)]
    FIXME
  \item[Konformität (usability compliance)]
    FIXME
\end{description}

\subsubsection{Effizienz (efficiency)}

\begin{description}
  \item[Zeitverhalten (time behaviour)]
    FIXME
  \item[Verbrauchsverhalten (resource utilization)]
    FIXME
  \item[Konformität (efficiency compliance)]
    FIXME
\end{description}

\subsubsection{Änderbarkeit (maintainability)}

\begin{description}
  \item[Analysierbarkeit (analyzability)]
    FIXME
  \item[Modifizierbarkeit (changeability)]
    FIXME
  \item[Stabilität (stability)]
    FIXME
  \item[Testbarkeit (testability)]
    FIXME
  \item[Konfromität (maintainability compliance)]
    FIXME
\end{description}

\subsubsection{Übertragbarkeit (portability)}

\begin{description}
  \item[Anpassbarkeit (adaptability)]
    FIXME
  \item[Installierbarkeit (installability)]
    FIXME
  \item[Koexistenz (co-existence)]
    FIXME
  \item[Austauschbarkeit (replaceability)]
    FIXME
  \item[Konformität (Portability compliance)]
    FIXME
\end{description}

\subsection{Schnittstellen}
%<Beschreibung der Schnittstellen der Software>

Folgende Schnittstellen werden im Projekt \emph{kitovu} zu Verfügung gestellt oder verwendet:

\begin{description}
  \item[Benutzeroberfläche]
    Die Benutzeroberfläche ist die Schnittstelle, mit welcher die meisten Endbenutzer mit der Applikation interagieren wird.
    Diese bietet einen einfachen Weg, um die Daten zu synchronisieren.
  \item[Command-Line-Programm]
    Alternativ bieten wir ein Command-Line-Programm, welches ermöglicht über einfache Befehle die Synchronisation zu starten.
    Dies ist besonders für Automatisierung oder für Endbenutzer, welche die Kommandozeile bevorzugen, gedacht.
  \item[Externe Schnittstellen]
    Wir verwenden als externe Schnittstellen das Studentenportal, das Moodle und den Skripteserver, sowie das Switch AAI für die Authorisierung des Moodles.
\end{description}

\subsection{Randbedingungen}
%<Auflistung der wichtigsten Randbedingungen mit einer Beschreibung dazu>

Dies technischen Randbedingungen wurden für das Projekt \emph{kitovu} festgelegt:

\begin{description}
  \item[Python] Version 3.6
  \item[Betriebssystem] Hauptsächlich Linux, aber auch Windows und macOS unterstützt
\end{description}

\end{document}
