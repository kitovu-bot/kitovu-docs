\documentclass[a4paper]{article}
\usepackage[margin=3cm]{geometry}
\usepackage[utf8]{inputenc}
\usepackage{cmbright}
\usepackage[hidelinks]{hyperref}
\usepackage{booktabs}
\usepackage[ngerman]{babel}
\usepackage{parskip}
\usepackage{graphicx}
\usepackage{minted}
\usepackage{pdflscape}
\usepackage{array}
\usepackage{tabulary}
\usepackage{multicol}
\usepackage{pgfgantt}
\usepackage{pgf-umlcd}
\usepackage{enumitem}
\usepackage{pifont}
\usepackage{threeparttable}

% Page breaks between sections
\let\oldsection\section
\renewcommand\section{\clearpage\oldsection}

% JIRA/Confluence shortcuts
\def\jiraurl{https://jira.keltec.ch/jira}
\def\confluenceurl{https://jira.keltec.ch/wiki}
\newcommand{\jiraissue}[1]{\href{\jiraurl/projects/EPJ/issues/EPJ-#1}{EPJ-#1}}
\newcommand{\fulljiraissue}[1]{EPJ-#1 (\url{\jiraurl/projects/EPJ/issues/EPJ-#1})}

% Tools
\newcommand{\tool}[2]{\emph{#1\footnote{\url{#2}}}}

\newcommand{\cmark}{\ding{51}}
\newcommand{\xmark}{\ding{55}}

\begin{document}
  \title{
    Projekt: kitovu \\
    \Large{Systemtests} \\[3em]
    \includegraphics[width=20em]{../../img/logo/kitovu.jpg}
  }
  \author{
    Florian Bruhin \\ \url{florian.bruhin@hsr.ch} \and
    Méline Sieber \\ \url{meline.sieber@hsr.ch} \and
    Nicolas Ganz \\ \url{nicolas.ganz@hsr.ch}
    }
  \date{\today}

  \maketitle

  \section*{Änderungsgeschichte}

  \begin{tabulary}{\linewidth}{llLl}
    \toprule
    Datum & Version & Änderung & AutorIn \\
    \midrule
    15.05.2018 & 1.0 & Dokument und Testprotokoll erstellt & Nicolas Ganz \\
    16.05.2018 & 1.1 & Testprotokoll überarbeitet und Performance Tests hinzugefügt & alle drei \\
    \bottomrule
  \end{tabulary}

  \pagebreak

  \section{Testprotokoll}

  Um die Benutzbarkeit von \emph{kitovu} zu prüfen, haben wir den Client verschiedenen Tests unterzogen. Zusätzlich zu separaten, automatisierten Tests haben wir hier Testprotokolle zusammengetragen, mit welchen wir komplette Abläufe getestet haben. Test 1 bis 3 erfolgten mit der Kommandozeilenversion von \emph{kitovu}, Test 4 mit der grafischen Oberfläche.

  Die Konfigurationsdateien sind am Ende des Kapitels beschrieben.

  \subsection{Test 1: Synchronisation}

  Dieser Test überprüft die grundsätzliche Funktionalität der Synchronisation.

  \subsubsection{Ablauf}

  \begin{enumerate}
    \item Dateien früherer Tests löschen.
    \item Konfiguration erstellen mit \verb|kitovu edit| und die Konfigurationsdatei 1 einfügen.
    \item Synchronisieren mit \verb|kitovu sync|.
    \item Erneut synchronisieren mit \verb|kitovu sync|.
  \end{enumerate}

  \subsubsection{Resultat}

  \begin{threeparttable}
    \begin{tabulary}{\linewidth}{lL|l}
      \toprule
      \textbf{Nr} & \textbf{Erwartet} & \textbf{\cmark / \xmark} \\
      \midrule
      1 & Sämtliche Dateien von EPJ und SE2 werden synchronisiert. & \cmark \\
      2 & Die Dateien wurden korrekt heruntergeladen und sind lesbar. & \cmark\tnote{1} \\
      3 & Beim SE2 werden die Daten von Moodle sowie vom Skripte-Server heruntergeladen. & \cmark \\
      4 & Beim EPJ werden die Daten nur vom Skripte-Server heruntergeladen. & \cmark \\
      5 & Während der ersten Synchronisation wird jede Datei ausgegeben, die heruntergeladen wird. & \cmark \\
      6 & Während der zweiten Synchronisation wird nur angezeigt, dass die Module synchronisiert werden. Es werden keine Dateinamen ausgegeben, da diese nicht heruntergeladen wurden. & \cmark \\
      \bottomrule
    \end{tabulary}

    \begin{tablenotes}
      \item[1] Unter anderem funktionieren PDF-Dateien einwandfrei.
    \end{tablenotes}
  \end{threeparttable}

  \subsection{Test 2: Abgebrochene Synchronisation}

  Dieser Test überprüft die Funktionalität der Synchronisation nach einem Abbruch.

  \subsubsection{Ablauf}

  \begin{enumerate}
    \item Dateien früherer Tests löschen.
    \item Konfiguration erstellen mit \verb|kitovu edit| und die Konfigurationsdatei 1 einfügen.
    \item Synchronisieren mit \verb|kitovu sync|, während der Ausführung das Programm mit \verb|CTRL+C| abbrechen.
    \item Erneut synchronisieren mit \verb|kitovu sync|.
  \end{enumerate}

  \subsubsection{Resultat}

  \begin{threeparttable}
    \begin{tabulary}{\linewidth}{lL|l}
      \toprule
      \textbf{Nr} & \textbf{Erwartet} & \textbf{\cmark / \xmark} \\
      \midrule
      1 & Während der ersten Synchronisation werden nur die ersten Dateien bis zum Abbruch heruntergeladen. & \cmark \\
      2 & Während der zweiten Synchronisation werden nur diejenigen Dateien heruntergeladen, die im ersten Durchgang fehlten. & \cmark \\
      3 & Die zweite Synchronisation zeigt keine Dateien an, welche bereits in der ersten Synchronisation heruntergeladen wurden. & \cmark \\
      \bottomrule
    \end{tabulary}

    %\begin{tablenotes}
    %  \item[1]
    %\end{tablenotes}
  \end{threeparttable}

  \subsection{Test 3: Ungültige Konfigurationsdatei}

  Dieser Test überprüft die Validierung der Konfigurationsdatei.

  \subsubsection{Ablauf}

  \begin{enumerate}
    \item Dateien früherer Tests löschen.
    \item Konfiguration erstellen mit \verb|kitovu edit| und die Konfiugrationsdatei 2 einfügen.
    \item Die Datei mit \verb|kitovu validate| überprüfen.
    \item Synchronisieren mit \verb|kitovu sync|.
  \end{enumerate}

  \subsubsection{Resultat}

  \begin{threeparttable}
    \begin{tabulary}{\linewidth}{lL|l}
      \toprule
      \textbf{Nr} & \textbf{Erwartet} & \textbf{\cmark / \xmark} \\
      \midrule
      1 & Die Validierung der Konfigurationsdatei führt zu einer Fehlermeldung, welche beschreibt, dass in der Skripte-Server-Connection der Eintrag \verb|plugin| fehlt. & \cmark\tnote{1} \\
      2 & Während der Synchronisation wird die gleiche Fehlermeldung angezeigt, es wird nichts synchronisiert. & \cmark \\
      \bottomrule
    \end{tabulary}

    \begin{tablenotes}
      \item[1] Das Problem wird grundsätzlich beschrieben, liesse sich aber noch klarer formulieren.
    \end{tablenotes}
  \end{threeparttable}

  \subsection{Test 4: Synchronisation über die GUI}

  Dieser Test überprüft die grundsätzliche Funktionalität der Synchronisation.

  \subsubsection{Ablauf}

  \begin{enumerate}
    \item Dateien früherer Tests löschen.
    \item GUI starten mit \verb|kitovu gui|
    \item Konfiguration erstellen über ``Konfiguration verwalten'', die Konfigurationsdatei 1 einfügen und über ``Speichern und zurück'' speichern.
    \item Synchronisieren mit ``Dateien synchronisieren''.
    \item Erneut synchronisieren mit ``Dateien synchronisieren''.
  \end{enumerate}

  \subsubsection{Resultat}

  \begin{threeparttable}
    \begin{tabulary}{\linewidth}{lL|l}
      \toprule
      \textbf{Nr} & \textbf{Erwartet} & \textbf{\cmark / \xmark} \\
      \midrule
      1 & Sämtliche Dateien von EPJ und SE2 werden synchronisiert. & \cmark \\
      2 & Die Dateien wurden korrekt heruntergeladen und sind lesbar. & \cmark\tnote{1} \\
      3 & Beim SE2 werden die Daten von Moodle sowie vom Skripte-Server heruntergeladen. & \cmark \\
      4 & Beim EPJ werden die Daten nur vom Skripte-Server heruntergeladen. & \cmark \\
	  5 & Während der ersten Synchronisation wird jede Datei ausgegeben, die heruntergeladen wird. & \cmark \\
	  6 & Während der zweiten Synchronisation wird nur angezeigt, dass die Module synchronisiert werden. Es werden keine Dateinamen ausgegeben, da diese nicht heruntergeladen wurden. & \cmark \\
      \bottomrule
    \end{tabulary}

    \begin{tablenotes}
      \item[1] Unter anderem funktionieren PDF-Dateien einwandfrei.
    \end{tablenotes}
  \end{threeparttable}

  \pagebreak

  \subsection{Konfigurationsdateien}

  \subsubsection{Konfigurationsdatei 1}

  \begin{minted}[
    gobble=4,
    frame=single,
  ]{yaml}
    root-dir: ~/Documents/HSR/Test/semester_06

    global-ignore:
      - Thumbs.db
      - .DS_Store

    connections:
      - name: skripte-server
        plugin: smb
        username: nganz
      - name: moodle
        plugin: moodle

    subjects:
      - name: Engineering Projekt
        sources:
        - connection: skripte-server
          remote-dir: Informatik/Fachbereich/Engineering-Projekt/EPJ
      - name: Software Engineering 2
        sources:
          - connection: moodle
            remote-dir: Software-Engineering 2 FS2018
          - connection: skripte-server
            remote-dir: Informatik/Fachbereich/Software-Engineering_2/SE2
  \end{minted}

  \subsubsection{Konfigurationsdatei 2}

  \begin{minted}[
    gobble=4,
    frame=single,
  ]{yaml}
    root-dir: ~/Documents/HSR/Test/semester_06

    connections:
      - name: skripte-server
        username: nganz

    subjects: []
  \end{minted}

  \section{Performance-Tests}

  Folgende Tests wurden im HSR-Netzwerk auf einem ``Lenovo ThinkPad''-Rechner unter Ubuntu 16.04 durchgeführt.

  \begin{threeparttable}
    \begin{tabulary}{\linewidth}{Llll}
      \toprule
      Modul & Anzahl Dateien & Sync 1 (min:s) \tnote{1} & Sync 2 (min:s) \tnote{2} \\
      \midrule
      Software Engineering 2 & 80 & 00:20 & 00:01 \\
      Web Engineering + Design & 651 & 00:55 & 00:15 \\
      Ganzes Semester (5 Module) & 1'086 & 01:35 & 00:13 \\
      \bottomrule
    \end{tabulary}

    \begin{tablenotes}
      \item[1] Synchronisation, bei der die Dateien noch nicht lokal existieren und heruntergeladen werden.
      \item[2] Synchronisation, bei der die Dateien bereits existieren.
    \end{tablenotes}
  \end{threeparttable}
\end{document}
