\documentclass[a4paper]{article}
\usepackage[margin=3cm]{geometry}
\usepackage[utf8]{inputenc}
\usepackage{cmbright}
\usepackage[hidelinks]{hyperref}
\usepackage{booktabs}
\usepackage[ngerman]{babel}
\usepackage{parskip}
\usepackage{graphicx}
\usepackage{minted}
\usepackage{pdflscape}
\usepackage{array}
\usepackage{tabulary}
\usepackage{multicol}
\usepackage{pgfgantt}
\usepackage{pgf-umlcd}
\usepackage{enumitem}
\usepackage{pifont}
\usepackage{threeparttable}

% Page breaks between sections
\let\oldsection\section
\renewcommand\section{\clearpage\oldsection}

% JIRA/Confluence shortcuts
\def\jiraurl{https://jira.keltec.ch/jira}
\def\confluenceurl{https://jira.keltec.ch/wiki}
\newcommand{\jiraissue}[1]{\href{\jiraurl/projects/EPJ/issues/EPJ-#1}{EPJ-#1}}
\newcommand{\fulljiraissue}[1]{EPJ-#1 (\url{\jiraurl/projects/EPJ/issues/EPJ-#1})}

% Tools
\newcommand{\tool}[2]{\emph{#1\footnote{\url{#2}}}}

\newcommand{\cmark}{\ding{51}}
\newcommand{\xmark}{\ding{55}}

\begin{document}
  \title{
    Projekt: kitovu \\
    \Large{Systemtests} \\[3em]
    \includegraphics[width=20em]{../../img/logo/kitovu.jpg}
  }
  \author{
    Florian Bruhin \\ \url{florian.bruhin@hsr.ch} \and
    Méline Sieber \\ \url{meline.sieber@hsr.ch} \and
    Nicolas Ganz \\ \url{nicolas.ganz@hsr.ch}
    }
  \date{\today}

  \maketitle

  \section*{Änderungsgeschichte}

  \begin{tabulary}{\linewidth}{llLl}
    \toprule
    Datum & Version & Änderung & AutorIn \\
    \midrule
    15.05.2018 & 1.0 & Dokument und Testprotokoll erstellt & Nicolas Ganz \\
    16.05.2018 & 1.1 & Testprotokoll überarbeitet und Performance Tests hinzugefügt & alle drei \\
    16.05.2018 & 1.2 & Usability Test Resultate übernommen & Nicolas Ganz \\
    \bottomrule
  \end{tabulary}

  \pagebreak

  \section{Usability Tests}
  
  Die Usability-Tests wurden ohne Referenz zur bereits bestehenden, externe Dokumentation durchgeführt.\footnote{\url{https://kitovu.readthedocs.io/}} Die Testpersonen haben nur mit den bestehenden Hinweisen gearbeitet, die das Programm selbst gibt.

  \subsection{Test 1}

  Dieser Test wurde mit einem Vollzeit-Informatikstudenten im 4. Semester durchgeführt und umfasst das Kommandozeilenprogramm sowie die graphische Benutzeroberfläche.

  \begin{itemize}
    \item \verb|kitovu sync| geht etwas unter bei all den Optionen, die \verb|--help| bietet.
    \item Die Angabe ``with the given configuration'' ist verwirrend, weil man ja normalerweise keine Datei angibt.
    \item ``Enter password for smb plugin'' -- was ist smb denn eigentlich?
    \item Hat die Synchronisation funktioniert? Abruptes Ende ohne Nachricht für eine erfolgreiche Synchronisation.
    \item YAML ist als Format für die Konfigurationsdatei etwas problematisch, vor allem bezüglich der exakt benötigten Einrückung.
    \item ``Synchronisation erfolgreich beendet'' wird angezeigt, auch wenn es Plugin-Fehler gab.
    \item Graphische Benutzeroberfläche: Tabulatortaste macht einen Tab anstatt 2 Spaces, was YAML erwartet.
    \item Graphische Benutzeroberfläche: Der Konfigurationseditor hat keine Zeilennummern. Diese würden bei Fehlermeldungen helfen.
    \item Graphische Benutzeroberfläche: Es ist nicht ganz klar, ob die Konfigurationsdatei erfolgreich gespeichert wurde.
  \end{itemize}

  \subsection{Test 2}

  Dieser Test wurde mit einer Studentin der Raumplanung im 2. Semester durchgeführt. Es wurde nur die graphische Benutzeroberfläche getestet und das Editieren der Konfiguration wurde grösstenteils ausgelassen.

  \begin{itemize}
    \item Mit ``Speichern'' passiert nichts, wenn die Konfiguration bearbeitet wurde. Sie erwartete, dass ``speichern'' automatisch zurück zum Startbildschirm führt. ``Speichern und zurück'' hat sie zuerst übersehen.
    \item Warum gibt es nach der Synchronisation einen ``Zurück''-Button?
    \item Der Zusammenhang zwischen Konfiguration und Synchronisation wurde nicht erkannt: Warum wählt man nach der Synchronisation nicht aus, wo die Dateien jetzt gespeichert werden sollen?
    \item Konzept der Konfiguration war zuerst unklar. Der ``Konfiguration editieren''-Knopf wurde zuerst nicht gefunden.
  \end{itemize}

  \subsection{Test 3}

  Dieser Test wurde mit einem Student der Raumplanung im 2. Semester durchgeführt. Auch hier wurde nur die graphische Benutzeroberfläche getestet, zusätzlich, wie die Konfigurationsdatei zu bearbeiten war, allerdings unter etwas Anleitung.

  \begin{itemize}
    \item Output grundsätzlich verständlich. Es würde helfen, die Module bzw. ``Sources''-Gruppen hervorzuheben.
    \item Die Aktionen waren alle klar, auch wie man sie findet.
    \item Das Editieren ist jedoch unverständlich, mit etwas Anleitung konnte er es selbständig durchführen.
  \end{itemize}

  \subsection{Test 4}

  Dieser Test wurde mit einem Teilzeit-Informatikstudent im 6. Semestern durchgeführt. Der Test umfasste das Kommandozeilenprogramm  sowie die graphische Benutzeroberfläche.

  \begin{itemize}
    \item Die YAML-Fehlermeldungen sind etwas kompliziert zu verstehen.
    \item Es wurde alles gefunden via \verb|--help| und \verb|[cmd] --help|.
    \item Der Log-Output ist verständlich.
    \item Graphische Benutzeroberfläche: Statt nach der Synchronisation ``Zurück'' zu drücken, wurde das zu verwendende Fenster geschlossen.
  \end{itemize}

  \subsection{Auswertung}

  Die genaue Auswertung haben wir in entsprechenden Jira-Issues zusammengefasst und mit dem Issue der Usability-Tests verknüpft.

  Das Issue ist unter \url{https://jira.keltec.ch/jira/browse/EPJ-23} zu finden.

  \section{Testprotokoll}

  Um die Benutzbarkeit von \emph{kitovu} zu prüfen, haben wir den Client verschiedenen Tests unterzogen. Zusätzlich zu separaten, automatisierten Tests haben wir hier Testprotokolle zusammengetragen, mit welchen wir komplette Abläufe getestet haben. Test 1 bis 3 erfolgten mit der Kommandozeilenversion von \emph{kitovu}, Test 4 mit der grafischen Oberfläche.

  Die Konfigurationsdateien sind am Ende des Kapitels beschrieben.

  \subsection{Test 1: Synchronisation}

  Dieser Test überprüft die grundsätzliche Funktionalität der Synchronisation.

  \subsubsection{Ablauf}

  \begin{enumerate}
    \item Dateien früherer Tests löschen.
    \item Konfiguration erstellen mit \verb|kitovu edit| und die Konfigurationsdatei 1 einfügen.
    \item Synchronisieren mit \verb|kitovu sync|.
    \item Erneut synchronisieren mit \verb|kitovu sync|.
  \end{enumerate}

  \subsubsection{Resultat}

  \begin{threeparttable}
    \begin{tabulary}{\linewidth}{lL|l}
      \toprule
      \textbf{Nr} & \textbf{Erwartet} & \textbf{\cmark / \xmark} \\
      \midrule
      1 & Sämtliche Dateien von EPJ und SE2 werden synchronisiert. & \cmark \\
      2 & Die Dateien wurden korrekt heruntergeladen und sind lesbar. & \cmark\tnote{1} \\
      3 & Beim SE2 werden die Daten von Moodle sowie vom Skripte-Server heruntergeladen. & \cmark \\
      4 & Beim EPJ werden die Daten nur vom Skripte-Server heruntergeladen. & \cmark \\
      5 & Während der ersten Synchronisation wird jede Datei ausgegeben, die heruntergeladen wird. & \cmark \\
      6 & Während der zweiten Synchronisation wird nur angezeigt, dass die Module synchronisiert werden. Es werden keine Dateinamen ausgegeben, da diese nicht heruntergeladen wurden. & \cmark \\
      \bottomrule
    \end{tabulary}

    \begin{tablenotes}
      \item[1] Unter anderem funktionieren PDF-Dateien einwandfrei.
    \end{tablenotes}
  \end{threeparttable}

  \subsection{Test 2: Abgebrochene Synchronisation}

  Dieser Test überprüft die Funktionalität der Synchronisation nach einem Abbruch.

  \subsubsection{Ablauf}

  \begin{enumerate}
    \item Dateien früherer Tests löschen.
    \item Konfiguration erstellen mit \verb|kitovu edit| und die Konfigurationsdatei 1 einfügen.
    \item Synchronisieren mit \verb|kitovu sync|, während der Ausführung das Programm mit \verb|CTRL+C| abbrechen.
    \item Erneut synchronisieren mit \verb|kitovu sync|.
  \end{enumerate}

  \subsubsection{Resultat}

  \begin{threeparttable}
    \begin{tabulary}{\linewidth}{lL|l}
      \toprule
      \textbf{Nr} & \textbf{Erwartet} & \textbf{\cmark / \xmark} \\
      \midrule
      1 & Während der ersten Synchronisation werden nur die ersten Dateien bis zum Abbruch heruntergeladen. & \cmark \\
      2 & Während der zweiten Synchronisation werden nur diejenigen Dateien heruntergeladen, die im ersten Durchgang fehlten. & \cmark \\
      3 & Die zweite Synchronisation zeigt keine Dateien an, welche bereits in der ersten Synchronisation heruntergeladen wurden. & \cmark \\
      \bottomrule
    \end{tabulary}

    %\begin{tablenotes}
    %  \item[1]
    %\end{tablenotes}
  \end{threeparttable}

  \subsection{Test 3: Ungültige Konfigurationsdatei}

  Dieser Test überprüft die Validierung der Konfigurationsdatei.

  \subsubsection{Ablauf}

  \begin{enumerate}
    \item Dateien früherer Tests löschen.
    \item Konfiguration erstellen mit \verb|kitovu edit| und die Konfiugrationsdatei 2 einfügen.
    \item Die Datei mit \verb|kitovu validate| überprüfen.
    \item Synchronisieren mit \verb|kitovu sync|.
  \end{enumerate}

  \subsubsection{Resultat}

  \begin{threeparttable}
    \begin{tabulary}{\linewidth}{lL|l}
      \toprule
      \textbf{Nr} & \textbf{Erwartet} & \textbf{\cmark / \xmark} \\
      \midrule
      1 & Die Validierung der Konfigurationsdatei führt zu einer Fehlermeldung, welche beschreibt, dass in der Skripte-Server-Connection der Eintrag \verb|plugin| fehlt. & \cmark\tnote{1} \\
      2 & Während der Synchronisation wird die gleiche Fehlermeldung angezeigt, es wird nichts synchronisiert. & \cmark \\
      \bottomrule
    \end{tabulary}

    \begin{tablenotes}
      \item[1] Das Problem wird grundsätzlich beschrieben, liesse sich aber noch klarer formulieren.
    \end{tablenotes}
  \end{threeparttable}

  \subsection{Test 4: Synchronisation über die GUI}

  Dieser Test überprüft die grundsätzliche Funktionalität der Synchronisation.

  \subsubsection{Ablauf}

  \begin{enumerate}
    \item Dateien früherer Tests löschen.
    \item GUI starten mit \verb|kitovu gui|
    \item Konfiguration erstellen über ``Konfiguration verwalten'', die Konfigurationsdatei 1 einfügen und über ``Speichern und zurück'' speichern.
    \item Synchronisieren mit ``Dateien synchronisieren''.
    \item Erneut synchronisieren mit ``Dateien synchronisieren''.
  \end{enumerate}

  \subsubsection{Resultat}

  \begin{threeparttable}
    \begin{tabulary}{\linewidth}{lL|l}
      \toprule
      \textbf{Nr} & \textbf{Erwartet} & \textbf{\cmark / \xmark} \\
      \midrule
      1 & Sämtliche Dateien von EPJ und SE2 werden synchronisiert. & \cmark \\
      2 & Die Dateien wurden korrekt heruntergeladen und sind lesbar. & \cmark\tnote{1} \\
      3 & Beim SE2 werden die Daten von Moodle sowie vom Skripte-Server heruntergeladen. & \cmark \\
      4 & Beim EPJ werden die Daten nur vom Skripte-Server heruntergeladen. & \cmark \\
	  5 & Während der ersten Synchronisation wird jede Datei ausgegeben, die heruntergeladen wird. & \cmark \\
	  6 & Während der zweiten Synchronisation wird nur angezeigt, dass die Module synchronisiert werden. Es werden keine Dateinamen ausgegeben, da diese nicht heruntergeladen wurden. & \cmark \\
      \bottomrule
    \end{tabulary}

    \begin{tablenotes}
      \item[1] Unter anderem funktionieren PDF-Dateien einwandfrei.
    \end{tablenotes}
  \end{threeparttable}

  \pagebreak

  \subsection{Konfigurationsdateien}

  \subsubsection{Konfigurationsdatei 1}

  \begin{minted}[
    gobble=4,
    frame=single,
  ]{yaml}
    root-dir: ~/Documents/HSR/Test/semester_06

    global-ignore:
      - Thumbs.db
      - .DS_Store

    connections:
      - name: skripte-server
        plugin: smb
        username: nganz
      - name: moodle
        plugin: moodle

    subjects:
      - name: Engineering Projekt
        sources:
        - connection: skripte-server
          remote-dir: Informatik/Fachbereich/Engineering-Projekt/EPJ
      - name: Software Engineering 2
        sources:
          - connection: moodle
            remote-dir: Software-Engineering 2 FS2018
          - connection: skripte-server
            remote-dir: Informatik/Fachbereich/Software-Engineering_2/SE2
  \end{minted}

  \subsubsection{Konfigurationsdatei 2}

  \begin{minted}[
    gobble=4,
    frame=single,
  ]{yaml}
    root-dir: ~/Documents/HSR/Test/semester_06

    connections:
      - name: skripte-server
        username: nganz

    subjects: []
  \end{minted}

  \section{Performance-Tests}

  Folgende Tests wurden im HSR-Netzwerk auf einem ``Lenovo ThinkPad''-Rechner unter Ubuntu 16.04 durchgeführt.

  \begin{threeparttable}
    \begin{tabulary}{\linewidth}{Llll}
      \toprule
      Modul & Anzahl Dateien & Sync 1 (min:s) \tnote{1} & Sync 2 (min:s) \tnote{2} \\
      \midrule
      Software Engineering 2 & 80 & 00:20 & 00:01 \\
      Web Engineering + Design & 651 & 00:55 & 00:15 \\
      Ganzes Semester (5 Module) & 1'086 & 01:35 & 00:13 \\
      \bottomrule
    \end{tabulary}

    \begin{tablenotes}
      \item[1] Synchronisation, bei der die Dateien noch nicht lokal existieren und heruntergeladen werden.
      \item[2] Synchronisation, bei der die Dateien bereits existieren.
    \end{tablenotes}
  \end{threeparttable}
\end{document}
