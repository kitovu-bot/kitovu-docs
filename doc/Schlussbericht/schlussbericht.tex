\documentclass[a4paper]{article}
\usepackage[margin=3cm]{geometry}
\usepackage[utf8]{inputenc}
\usepackage{cmbright}
\usepackage[hidelinks]{hyperref}
\usepackage{booktabs}
\usepackage[ngerman]{babel}
\usepackage{parskip}
\usepackage{graphicx}
\usepackage{minted}
\usepackage{pdflscape}
\usepackage{array}
\usepackage{tabulary}
\usepackage{multicol}
\usepackage{pgfgantt}
\usepackage{pgf-umlcd}
\usepackage{enumitem}
\usepackage{pifont}
\usepackage{threeparttable}

% Page breaks between sections
\let\oldsection\section
\renewcommand\section{\clearpage\oldsection}

% JIRA/Confluence shortcuts
\def\jiraurl{https://jira.keltec.ch/jira}
\def\confluenceurl{https://jira.keltec.ch/wiki}
\newcommand{\jiraissue}[1]{\href{\jiraurl/projects/EPJ/issues/EPJ-#1}{EPJ-#1}}
\newcommand{\fulljiraissue}[1]{EPJ-#1 (\url{\jiraurl/projects/EPJ/issues/EPJ-#1})}

% Tools
\newcommand{\tool}[2]{\emph{#1\footnote{\url{#2}}}}

\newcommand{\cmark}{\ding{51}}
\newcommand{\xmark}{\ding{55}}

\begin{document}
  \title{
    Projekt: kitovu \\
    \Large{Schlussbericht} \\[3em]
    \includegraphics[width=20em]{../../img/logo/kitovu.jpg}
  }
  \author{
    Florian Bruhin \\ \url{florian.bruhin@hsr.ch} \and
    Méline Sieber \\ \url{meline.sieber@hsr.ch} \and
    Nicolas Ganz \\ \url{nicolas.ganz@hsr.ch}
    }
  \date{\today}

  \maketitle

  \section*{Änderungsgeschichte}

  \begin{tabulary}{\linewidth}{llLl}
    \toprule
    Datum & Version & Änderung & AutorIn \\
    \midrule
    16.05.2018 & 1.0 & Dokument erstellt und Struktur aus der Vorlage übernommen & Nicolas Ganz \\
    \bottomrule
  \end{tabulary}

  \pagebreak

  \section{Zielerreichung}

  % FIXME: <Objektive Zielerreichung beschreiben>

  \section{Allgemeiner Erfahrungsbericht}

  % FIXME: <Allgemeine Erfahrungen, welche als Team gesammelt wurden (z.B. in Meetings, beim Auftreten von Problemen, usw.)>

  \section{Persönliche Erfahrungen}

  \subsection{Méline Sieber}

Nach diesem Engineering-Projekt weiss ich: Es gibt Einhörner. 

Gruppenarbeitseinhörner. Solche Team-Arbeiten, in denen einfach alles rund und perfekt läuft. Ich staune noch jetzt, wie reibungslos alles lief. Als Quereinsteigerin war für mich so gut wie alles neu, so gut wie alles machte ich zum ersten Mal: Werkzeuge wie Jira, Github, Travis verwenden; ein Programm von Grund auf schreiben, also mehr als nur einen fehlenden Programmausschnitt ergänzen; im Team Software entwickeln.

Der Einstieg war erwartungsgemäss steil, das Wasser kalt. Viele Konzepte aus der Elaborationsphase verstand ich erst in der Anwendung (Travis, AppVeyor). Erst nach etwa zwei Sprints erhielt ich langsam ein Gefühl für die Arbeitsabläufe (Pull Requests, Code Reviews etc.). Glück im Unglück war der FileCache: Aufgrund krankheitsbedingter Abwesenheit von Florian fiel mir dieser anspruchsvolle Teil von kitovu zu. Dank dieser Aufgabe erreichte ich endlich die benötigte Programmierroutine. Des Weiteren wurde mir klar, wie wichtig Tests sind -- denn anhand dieser verstand ich nebenbei, wie Code funktionierte, den ich nicht selber geschrieben habe.

Ein unverzichtbarer, äusserst wertvoller Teil des Lernprozesses war das Pair-Programming mit Florian und Nicolas. Beide nahmen sich viel Zeit, meine Hürden gemeinsam zu nehmen. Das senkte die Berührungsängste vor der für mich anspruchsvollen Aufgabe und gab mir die nötige Denkroutine. Hier auch ein grosser Dank an die beiden: Sie beantworteten geduldig jegliche Fragen. Trotz meiner mangelnden Erfahrung habe ich mich nie als unnützes Rad am Projektwagen gefühlt, ich wurde in alle Diskussionen mit einbezogen, meine Meinung hatte gleiches Gewicht wie das der anderen. Eine weitere, wichtige Erkenntnis war wohl, dass auch erfahrene Programmierer nur mit Wasser kochen. Wo ich selber die meisten Probleme hatte, lagen meist die allgemeinen Knacknüsse des Projekts, auch für Florian und Nicolas.

Die weniger technische Seite des Projekts verlief ebenfalls wunschgemäss. Die wichtigste Entscheidung haben wir schon zu Beginn getroffen, nämlich auf ein weiteres Team-Mitglied zu verzichten. Die Dreiergruppe passte zum Projektumfang, da wir von Anfang an das Projekt klein, aber ausbaubar gehalten haben. Das hat sich ausbezahlt: Wir haben sowohl das Kernziel erfüllt (Skripteserver-Einbindung) als auch die Optionen (Moodle-Einbindung). Dass Moodle geklappt hat, war wohl meine grösste Überraschung -- ich hatte mit viel mehr Hindernissen gerechnet.

SE1 betonte, wie wichtig die physische Nähe der Teammitglieder ist -- das hat sich bestätigt. Die virtuelle Kommunikation via Telegram war eine gute Unterstützung, aber die wöchentlichen Sitzungen haben sich mehr als ausbezahlt. Oft merkten wir erst dann, dass wir nicht vom Gleichen sprachen und konnten uns so auf übereinstimmende Bedeutungen einigen. Das wurde vor allem in Bezug auf die Fehlerbehandlung klar: Wir hatten uns nicht in einer solchen Sitzung abgesprochen, sondern nur über Telegram. Das führte zu einem Missverständnis und einigem Refactoring, was aber zum Glück keine grossen Auswirkungen hatte -- eine Lehre war es allemal.

Textualität ist auch bei Softwareprojekten wichtig, stellte ich erfreut fest. Wir haben mehrere Dokumente verfasst (Projektplan, Anforderungen, Architektur), die auch intern sehr nützlich waren und die wir häufig referenzierten. Erst wenn es schriftlich ist, ist es definitiv: Das half, die Dinge zu Ende zu denken.

Kurzum, das Engineering Projekt kann ich mit der Erkenntnis abschliessen: Einhörner sind möglich.

  \subsection{Florian Bruhin}

  % FIXME

  \subsection{Nicolas Ganz}

  \subsubsection{Umstellung zu Python}

  Ich bin mir Ruby gewohnt, was grundsätzlich ähnlich ist, sich aber doch in diversen Kleinigkeiten unterscheidet.
  Für mich ist es einfacher auf eine Sprache umzustellen, welche komplett anders ist, wie zum Beispiel Java.
  Da weiss ich, dass es etwas komplett anderes ist, aber bei Python vergleiche ich es fast schon zu fest mit Ruby.

  Die Python Guidelines waren für mich daher zu Beginn mühsam.
  Da die Tools mich aber immer wieder an die Guidelines erinnerten, konnte ich mir diese besser merken.
  Mittlerweilen sind sie mir auch ziemlich geläufig und ich konnte meinen Fortschritt von Pull Request zu Pull Request bemerken.

  \subsubsection{Kommunikation im Team}

  Die Zusammenarbeit im Team war sehr dynamisch und verlief einwandfrei.
  Das Feedback und die Diskussionen waren immer konstruktiv und im Sinne des Projektes.

  Besonders geschätzt habe ich die Diskussionen über Konzeptentscheide.
  Sämtliche Teammitglieder haben sich daran aktiv beteiligt und sind kompromissbereit gewesen.
  Dadurch haben wir einen soliden und sauberen Grundstein für einen neuen Synchronisations-Client gelegt.

  Auch die Code-Reviews über die Github Pull Requests waren sehr nützlich und der Code wurde dabei immer gründliche geprüft.
  Davon konnte ich sehr profitieren, da Python für mich noch nicht so gängig ist und so diverse Kleinigkeiten hervorkamen, welche wir optimieren konnten.

  \subsubsection{Projektidee}

  Da ich dieses Projekt zukünftig verwenden will, war meine Motivation sehr hoch.
  Es hat mir auch die Motivation gegeben, diese Applikation so umzusetzen, dass sie in Zukunft wartbar und erweiterbar ist.

  Es war auch sehr motivierend bei den Usability-Tests zu erkennen, dass ein Bedürfnissfür eine solche Applikation auch studiengangsübergreifend vorhanden ist.

  \subsubsection{Fazit}

  Alles im Allem war dieses Projekt für mich ein grosser Erfolg.
  Ich konnte stark profitieren, habe diverses gelernt und vor allem hat es wirklich Spass gemacht.

  Sicherlich werde ich mich an der Weiterentwicklung des Projektes auch ausserhalb des Engineering Projektes beteiligen.

\end{document}
