\documentclass[a4paper]{article}
\usepackage[margin=3cm]{geometry}
\usepackage[utf8]{inputenc}
\usepackage{cmbright}
\usepackage[hidelinks]{hyperref}
\usepackage{booktabs}
\usepackage[ngerman]{babel}
\usepackage{parskip}
\usepackage{graphicx}
\usepackage{minted}
\usepackage{pdflscape}
\usepackage{array}
\usepackage{tabulary}
\usepackage{multicol}
\usepackage{pgfgantt}
\usepackage{pgf-umlcd}
\usepackage{enumitem}
\usepackage{pifont}
\usepackage{threeparttable}

% Page breaks between sections
\let\oldsection\section
\renewcommand\section{\clearpage\oldsection}

% JIRA/Confluence shortcuts
\def\jiraurl{https://jira.keltec.ch/jira}
\def\confluenceurl{https://jira.keltec.ch/wiki}
\newcommand{\jiraissue}[1]{\href{\jiraurl/projects/EPJ/issues/EPJ-#1}{EPJ-#1}}
\newcommand{\fulljiraissue}[1]{EPJ-#1 (\url{\jiraurl/projects/EPJ/issues/EPJ-#1})}

% Tools
\newcommand{\tool}[2]{\emph{#1\footnote{\url{#2}}}}

\newcommand{\cmark}{\ding{51}}
\newcommand{\xmark}{\ding{55}}

\begin{document}
  \title{
    Projekt: kitovu \\
    \Large{Schlussbericht} \\[3em]
    \includegraphics[width=20em]{../../img/logo/kitovu.jpg}
  }
  \author{
    Florian Bruhin \\ \url{florian.bruhin@hsr.ch} \and
    Méline Sieber \\ \url{meline.sieber@hsr.ch} \and
    Nicolas Ganz \\ \url{nicolas.ganz@hsr.ch}
    }
  \date{\today}

  \maketitle

  \section*{Änderungsgeschichte}

  \begin{tabulary}{\linewidth}{llLl}
    \toprule
    Datum & Version & Änderung & AutorIn \\
    \midrule
    16.05.2018 & 1.0 & Dokument erstellt und Struktur aus der Vorlage übernommen & Nicolas Ganz \\
    \bottomrule
  \end{tabulary}

  \pagebreak

  \section{Zielerreichung}

  % FIXME: <Objektive Zielerreichung beschreiben>

  \section{Allgemeiner Erfahrungsbericht}

  % FIXME: <Allgemeine Erfahrungen, welche als Team gesammelt wurden (z.B. in Meetings, beim Auftreten von Problemen, usw.)>

  \section{Persönliche Erfahrungen}

  \subsection{Méline Sieber}

  % FIXME

  \subsection{Florian Bruhin}

  % FIXME

  \subsection{Nicolas Ganz}

  \subsubsection{Umstellung zu Python}

  Ich bin mir Ruby gewohnt, was grundsätzlich ähnlich ist, sich aber doch in diversen Kleinigkeiten unterscheidet.
  Für mich ist es einfacher auf eine Sprache umzustellen, welche komplett anders ist, wie zum Beispiel Java.
  Da weiss ich dass es etwas komplett anderes ist, aber bei Python vergleiche ich es fast schon zu fest mit Ruby.

  Die Python Guidelines waren für mich daher zu Beginn mühsam.
  Da die Tools mich aber immer wieder an die Guidelines erinnerten konnte ich mir diese besser merken.
  Mittlerweilen sind sie mir auch ziemlich geläufig und ich konnte den Fortschritt von Pull Request zu Pull Request bemerken.

  \subsubsection{Kommunikation im Team}

  Die Zusammenarbeit im Team war sehr dynamisch und verlief einwandfrei.
  Das Feedback und die Diskussionen waren immer konstruktiv und im Sinne des Projektes.

  Besonders geschätzt habe ich die Diskusionen über Konzeptentscheide.
  Sämtliche Teammitglieder haben sich daran aktiv beteiligt und sind kompromissbereit gewesen.
  Dadurch haben wir einen soliden und sauberen Grundstein für einen neuen Synchronisations-Client gelegt.

  Auch die Code-Reviews über die Github Pull Requests waren sehr nützlich und der Code wurde dabei immer gründliche geprüft.
  Davon konnte ich sehr profitieren, da Python für mich noch nicht so gängig ist und so diverse Kleinigkeiten hervor kamen, welche wir optimieren konnten.

  \subsubsection{Projektidee}

  Da ich dieses Projekt zukünftig verwenden will war meine Motivation sehr hoch.
  Es hat mir auch die Motivation gegeben, diese Applikation so umzusetzen, dass sie in Zukunft wartbar und erweiterbar ist.

  Es war auch sehr motivierend bei den Usability Tests zu erkennen, dass ein Bedürfniss für eine solche Applikation auch studiengangübergreiffend vorhanden ist.

  \subsubsection{Fazit}

  Alles im Allem war dieses Projekt für mich ein grosser Erfolg.
  Ich konnte starkt profitieren, habe diverses gelernt und vorallem hat es wirklich Spass gemacht.

  Sicherlich werde ich mich an der Weiterentwicklung des Projektes auch ausserhalb des Engineering Projektes beteiligen.

\end{document}
