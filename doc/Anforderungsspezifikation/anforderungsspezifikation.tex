\documentclass[a4paper]{article}
\usepackage[margin=3cm]{geometry}
\usepackage[utf8]{inputenc}
\usepackage{cmbright}
\usepackage{hyperref}
\usepackage{booktabs}
\usepackage[ngerman]{babel}
\usepackage{parskip}
\usepackage{graphicx}
\usepackage{minted}
\usepackage{pdflscape}
\usepackage{array}
\usepackage{tabulary}
\usepackage{multicol}
\usepackage{pgfgantt}
\usepackage{pgf-umlcd}

% Page breaks between sections
\let\oldsection\section
\renewcommand\section{\clearpage\oldsection}

% JIRA/Confluence shortcuts
\def\jiraurl{https://jira.keltec.ch/jira}
\def\confluenceurl{https://jira.keltec.ch/wiki}
\newcommand{\jiraissue}[1]{\href{\jiraurl/projects/EPJ/issues/EPJ-#1}{EPJ-#1}}
\newcommand{\fulljiraissue}[1]{EPJ-#1 (\url{\jiraurl/projects/EPJ/issues/EPJ-#1})}

% Tools
\newcommand{\tool}[2]{\emph{#1\footnote{\url{#2}}}}

\begin{document}
	\title{
		Projekt: kitovu \\
		\Large{Anforderungsspezifikation} \\[3em]
		\includegraphics[width=20em]{../../img/logo/kitovu.jpg}
	}
	\author{
		Florian Bruhin \\ \url{florian.bruhin@hsr.ch} \and
		Méline Sieber \\ \url{meline.sieber@hsr.ch} \and
		Nicolas Ganz \\ \url{nicolas.ganz@hsr.ch} 
		}
	\date{\today}
	
	\maketitle

\section*{Änderungsgeschichte}

\begin{tabulary}{\linewidth}{llLl}
	\toprule
	Datum & Version & Änderung & AutorIn \\
	\midrule
	04.03.2018 & 1.0 & Dokument erstellt, Grundgerüst von Template übernommen, funktionale Anforderungen & Méline Sieber \\
	06.03.2018 & 1.1 & Überarbeitung & alle \\
	\bottomrule
\end{tabulary}
\pagebreak

\section{Einführung}
Dieses Dokument beschreibt, was \emph{kitovu} genau ist. Danach veranschaulicht diese Anforderungsspezifikation, wer \emph{kitovu} verwendet. Das erfolgt anhand von ``Use Cases'', einerseits in einem kurzen Format (\emph{brief}), andererseits in einer ausführlichen Beschreibung (\emph{fully dressed}). Weitere Anforderungen detaillieren, welche Qualitätsmerkmale und Schnittstellen verwendet werden sowie geltende Randbedingungen.

\subsection{Gültigkeitsbereich}
Die vorliegende Anforderungsspezifikation ist für das Engineering-Projekt im Frühlingssemester 2018 gültig. Falls dem Projekt grössere Veränderungen widerfahren, wird das Dokument dementsprechend angepasst. Umfassende Änderungen werden am Anfang des Dokuments protokolliert.

\subsection{Referenzen}

% <Liste aller verwendeten und referenzierten Dokumente, Bücher, Links, usw.>
% <Referenz auf ein Glossar Dokument, wo alle Abkürzungen und unklaren Begriffe erklärt werden>
% <Die Quellen / Referenzen sollten mit dem Word Tool automatisch erstellt werden>

Die Anforderungsspezifikation ist eng mit der Domainanalyse und anderen Dokumenten verbunden. Die folgende Tabelle listet die wichtigsten Referenzen auf.

\begin{tabulary}{\linewidth}{ll}
	Confluence & \url{\confluenceurl} \\
	Domainanalyse & FIXME \\
	Draw.io & \url{https://www.draw.io/} \\
	Github-Repository von \emph{kitovu} & \url{https://github.com/kitovu-bot/kitovu} \\
	JIRA	& \url{\jiraurl} \\
	Moodle & \url{https://moodle.hsr.ch} \\
	OpenHSR Connect & \url{https://github.com/openhsr/connect} \\
	Studentenportal & \url{https://studentenportal.ch/} \\
	Switch AAI (Authentication and Authorization Infrastructure)& \url{https://www.switch.ch/aai/} \\
	
\end{tabulary}

Beim Logo auf der Titelseite handelt es sich um eine stark überarbeitete Version eines GIFs (\url{https://www.animateit.net/details.php?image_id=8990}). Urheber und Copyright waren nicht auffindbar.

\pagebreak
\section{Allgemeine Beschreibung}

\subsection{Produktperspektive}
%<Produkt Perspektive beschreiben>

\subsection{Produktfunktion}
%<Allgemeine Beschreibung der Funktionen>
\emph{Kitovu} ist ein Client, der von verschiedenen Plattformen ausgewählte HSR-Unterrichtsmaterialien auf den eigenen Rechner synchronisiert. Er läuft auf allen gängigen Betriebssystemen und funktioniert nicht nur für den HSR-Skripteserver, sondern ist auch erweiterbar für verschiedene Plattformen.

Unser Projekt bindet primär den Skripteserver ein. Der Kommandozeilen-basierte Client funktioniert mittels Profilen zu unterschiedlichen Plattformen (Moodle, Skripteserver, Studentenportal). Pro Profil sind Verbindungsdaten und eventuelle Login-Credentials im Client hinterlegt. Die Daten-Synchronisation erfolgt immer nur von Server zu Client, erfolgreiche und misslungene Datentransfers werden protokolliert. Ein rudimentäres GUI dient als Proof-of-Concept.

Pro Profil lässt sich Folgendes definieren:

\begin{itemize}
	\item welche Ordner/Dateien synchronisiert werden.
	\item welche Ordner/Dateien von der Synchronisation ausgeschlossen werden.
	\item wie mit Duplikaten/lokal bestehenden Dateien umgegangen wird.
\end{itemize}

\emph{Kitovu} ist ausbaubar und damit modular: Zusätzlich zu den beiden Plattformen (Skripteserver; Moodle oder Studentenportal) können in zukünftigen Projekten beliebig viele Plattformen als separates Plugin bzw. Profil realisiert werden.

Optionale Features:

\begin{itemize}
	\item Moodle und/oder das Studentenportal.\footnote{Die Implementation von Moodle oder des Studentenportals ist abhängig von verschiedenen Risiken, die der Projektplan genauer ausführt.} 
	\item Komplettes GUI, das der Funktionalität des Kommandozeilen-Clients entspricht.
\end{itemize}

\subsection{Benutzercharakteristik}
%<Zielgruppe des Produktes>
Studentinnen und Studenten verwenden \emph{kitovu}, um ihre Unterrichtsmaterialien auf ihren Rechnern à jour zu halten. Sie verwenden verschiedene Betriebssysteme (Windows, macOS, Linux). Ihre Erfahrung mit der Kommandozeile ist unterschiedlich; manche verwenden sie nie, andere benutzen sie zur Standardinteraktion mit ihrem Betriebssystem.

Dozentinnen und Dozenten können ebenfalls \emph{kitovu} verwenden, sie sind jedoch nicht die primäre Zielgruppe.

\subsection{Einschränkungen}
%<Wo sind die Grenzen des Produkts>
Da die Projekt-Zeitspanne kurz ist, ist die Kernfunktion von \emph{kitovu} ein Client, der über die Kommandozeile bedient wird. Es ist vorerst nur eine rudimentäre grafische Benutzeroberfläche geplant. Falls jedoch genügend Zeit bleibt, baut das Team die Benutzeroberfläche aus. Uns ist bewusst, dass wir damit einen Teil der Studierenden ausschliessen, nicht alle können mit der Kommandozeile umgehen. Aufgrund der Zeitbeschränkung müssen wir das in Kauf nehmen, sehen es aber als erste Priorität. Falls genügend Zeit bleiben sollte, erweitern wir den grafischen Client auf die Funktionalität des Terminal-Clients.

Weitere Einschränkungen sind das Studentenportal und Moodle. Die Gründe dazu beschreibt bereits der Projektplan ausführlich.


\subsection{Annahmen}
%<Was ist unklar und wird angenommen bezüglich des Projektes oder des Produktes>
Eine wichtige Voraussetzung sind bereits bestehende Accounts. Wir gehen davon aus, dass die Studierenden bereits einen HSR-Account besitzen und sich damit sowohl per VPN von zu Hause als auch direkt an der HSR mit dem Skripteserver verbinden können. Das Studentenportal\footnote{\url{https://studentenportal.ch/}} verlangt einen separaten Account, der Zugriff auf Moodle\footnote{\url{https://moodle.hsr.ch}} erfolgt via Switch-AAI, der Authentisierung- und Autorisierungsschnittstelle für alle Schweizer Hochschulen\footnote{\url{https://www.switch.ch/aai/}}.

Wir müssen annehmen, dass ein Teil der Studentinnen und Studenten damit vertraut ist, die Kommandozeile zu bedienen. Die Konfiguration von \emph{kitovu} erfolgt über ein einzelnes Konfigurationsfile -- wir müssen ebenfalls davon ausgehen, dass die Studierenden damit umgehen können. Folglich schliesst ein Kommandozeilen-basierter Client einen Teil der Studierenden aus.

\subsection{Abhängigkeiten}
%<Von welchen Faktoren hängt das Produkt ab>
\emph{Kitovu} steht und fällt mit der Anbindung an die Plattformen, also den HSR-Skripteserver sowie die optionalen Plattformen Moodle oder Studentenportal.


\pagebreak
\begin{landscape}
  \thispagestyle{empty}
  \section{Domainanalyse}
  
  FIXME: schreib was hin
  
  \subsection{Domainmodell}
  
  % https://github.com/xuyuan/pgf-umlcd/blob/master/pgf-umlcd-manual.pdf
  \begin{tikzpicture}
    \begin{class}{File}{0,0}
      \attribute{path}
      \attribute{localDigest}
      \attribute{remoteDigest}
      \attribute{excluded : bool}
    \end{class}
  
    \begin{class}{AbstractSyncPlugin}{9,-0.4}
      \attribute{name : string}
      \attribute{version}
    \end{class}

    \association{File}{}{0..*}{AbstractSyncPlugin}{0..1}{}
  
    \begin{class}{MoodleSyncPlugin}{9, -5}
      \inherit{AbstractSyncPlugin}
    \end{class}
  
    \begin{class}{StudentenportalSyncPlugin}{6, -3}
      \inherit{AbstractSyncPlugin}
    \end{class}
  
    \begin{class}{SkripteserverSyncPlugin}{12, -3}
      \inherit{AbstractSyncPlugin}
    \end{class}

    \begin{class}{Profile}{14,2}
    \end{class}

    \begin{class}{Configuration}{19,0}
    \end{class}

    \association{Profile}{}{1}{Configuration}{1}{}
    \association{Profile}{}{1}{AbstractSyncPlugin}{1}{}

  \end{tikzpicture}
\end{landscape}

\pagebreak
\section{Funktionale Anforderungen}

\subsection{Aktoren und Stakeholder}
%<Aufzählung und Beschreibung der Aktoren & Stakeholder>

\begin{tabulary}{\linewidth}{lL}
	\toprule
	Aktor & Beschreibung\\
	\midrule
	Studentin & HSR-Studentin, die über einen HSR-Account verfügt.\\
	Skripteserver & HSR-Plattform, in der Dozenten Unterrichtsmaterialien zur Verfügung stellen.\\
	Studentenportal & Portal für eigene Materialien der HSR-Studierenden.\\
	Switch-AAI & Supporting Actor; ermöglicht Zugriff auf Moodle \\	
	Moodle & Zugriff nur via Switch-AAI-Token\\
	
	\bottomrule
\end{tabulary}

\subsection{Use-Case-Diagramm}
%<Use Case Diagramm>

FIXME insert definitive use case diagram, schön anordnen

\includegraphics[width=40em]{../../doc/Anforderungsspezifikation/uc_diagram_kitovu.png}

\subsection{Beschreibungen (Brief)}
%<Alle Use Cases in einzelnen Kapiteln beschreiben im Brief Format>
\begin{description}
	
\item[Use Case 1: Konfiguration editieren:] Die Studentin legt in einer Konfigurationsdatei alle Plattformen und die entsprechenden Verbindungsdaten fest, mit der auf die externen Plattformen zugegriffen wird.

\item[Use Case 2: Dateien auswählen:] Die Studentin legt in der Konfigurationsdatei fest, welche Ordner und Dateien synchronisiert werden sollen.

\item[Use Case 3: Verbindung erstellen:] Die Studentin stellt eine Verbindung zur externen Plattform her und authentisiert sich dort.

\item[Use Case 4: Synchronisation:] Die Studentin synchronisiert die Dateien von der externen Plattform auf ihren Rechner.
\end{description}

\pagebreak
\subsection{Beschreibungen (Fully Dressed)}
%<Spezielle und wichtige Use Cases in einzelnen Kapiteln beschreiben im Fully Dressed Format>

FIXME Nicolas Tabelle in schön- Weggelassen: Level, weil macht irgendwie keinen Sinn
FIXME Méline: Use Cases umschreiben, Brief-Version folgend. Überall Studentin

\subsubsection{UC1: Konfiguration editieren}
\begin{tabulary}{\linewidth}{lp{30em}}
	\textbf{Goal} & Der Student editiert die Konfiguration, in der alle externen Plattform enthalten ist. \\
	%\textbf{Level} & FIXME \\
	\textbf{Primary Actor} & Student\\
	\textbf{Trigger} & Der Student möchte aktuelle Unterrichtsmaterialien auf seinem Rechner haben. \\
	\textbf{Stakeholders and Interests} & \textbf{Der Student}: legt eine neue Plattform an. \textbf{Das System:} informiert den Studenten, welche Plattformen unterstützt werden und stellt sicher, dass die Plattform bereits implementiert ist. \\
	\textbf{Preconditions} & Der Student besitzt einen Account der externen Plattform mit entsprechenden Login-Daten.\\
	\textbf{Postconditions} & Die aktualisierte Konfigurationsdatei wird gespeichert. \\
	\textbf{Main Success Scenario} & Der Student hinterlegt in der Konfiguration von \emph{kitovu} zu einer externen Plattform in \emph{kitovu}. Er kann die Zugangsdaten erstellen, speichern, abändern oder das Profil ganz entfernen. Die eingegebenen Daten werden in \emph{kitovu} gespeichert. \\
	\textbf{Extensions} & Die eingegebenen Daten sind ungültig. \emph{Kitovu} weist die Studentin darauf hin.\\
	%\textbf{Special Requirements} & FIXME \\
	\textbf{Frequency of Occurrence} & Selten, nach Bedarf. \\
\end{tabulary}

\subsubsection{UC2: Dateien auswählen}
\begin{tabulary}{\linewidth}{lp{30em}}
	\textbf{Goal} & Die Studentin schreibt in die Konfigurationsdatei, welche Dateien sie auf ihren Rechner synchronisieren möchte. Sie kann ausschliessen, welche Dateien nicht synchronisiert werden.\\
	%\textbf{Level} & FIXME \\
	\textbf{Primary Actor} & Studentin\\
	\textbf{Trigger} & Die Studentin möchte nur bestimmte Dateien und Ordner auf ihrem Rechner haben.\\
	\textbf{Stakeholders and Interests} & \textbf{Die Studentin}: weiss, welche Dateien sie synchronisieren möchte. \textbf{Das System:} kann die entsprechenden Dateien lokalisieren. \\
	\textbf{Preconditions} & UC1 ist erfolgt: Die Studentin besitzt einen Account der externen Plattform. Sie hat bereits ein Profil der externen Plattform in \emph{kitovu} hinterlegt. [FIXME linebreak] Sie weiss, welche Dateien sie synchronisieren möchte und wo sie auf der externen Plattform abgelegt sind. \\
	\textbf{Postconditions} & Die aktualisierte Konfigurationsdatei wird gespeichert.\\
	\textbf{Main Success Scenario} & Die Studentin definiert für das jeweilige Profil, welche Dateien sie synchronisiert haben möchte und welche ausgeschlossen werden. \\
	\textbf{Extensions} & Die angegebenen Pfade existieren nicht -- die Studentin wird in UC4 darauf hingewiesen.\\
	%\textbf{Special Requirements} & asdf \\
	\textbf{Frequency of Occurrence} & Zu Beginn jedes Semesters, wenn die neuen Module beginnen. \\
	%\textbf{Open Issues} & asdf \\
\end{tabulary}

\subsubsection{UC3: Verbindung erstellen}
\begin{tabulary}{\linewidth}{lp{30em}}
	\textbf{Goal} & Um auf die Unterrichtsmaterialien zuzugreifen, muss der Student eine Verbindung zur externen Plattform herstellen. \\
	%\textbf{Level} & FIXME \\
	\textbf{Primary Actor} & Der Student \\
	\textbf{Trigger} & Der Student möchte auf eine externe Plattform zugreifen. \\
	\textbf{Stakeholders and Interests} & \textbf{Der Student}: weiss, von welcher Plattform er welche Dateien möchte. \textbf{Das System:} kann die Verbindung herstellen. \\
	\textbf{Preconditions} & Der Student hat in UC1 die Profildaten der externen Plattform hinterlegt.\\
	\textbf{Postconditions} & Keine.\\
	\textbf{Main Success Scenario} & Der Student verbindet sich zur externen Plattform, die Authentisierung ist erfolgreich. \\
	\textbf{Extensions} & \textbf{Die Verbindung misslingt:} [FIXME Bulletpoint] Nicht im HSR-Netz (HSR-Skripteserver): Die Verbindung misslingt, da sich der Student nicht im HSR-Netz befindet oder nicht per VPN verbunden ist. [FIXME Bulletpoint] Moodle-Probleme: der Zugriff auf Moodle misslingt aufgrund Plattform-eigener Probleme. [FIXME Bulletpont] Die hinterlegten Login-Daten sind nicht korrekt.
	\textbf{Keine Login-Daten hinterlegt:} \emph{kitovu} fordert die Studentin auf, ihre Login-Daten einzugeben.
	\textbf{Moodle:} Die externe Plattform verlangt, dass sich der Student pro Session authentisiert. \textbf{HSR-Skripteserver:} Die externe Plattform verlangt, dass sich der Student im HSR-Netz befindet oder bereits eine VPN-Verbindung zur Plattform aufgenommen hat.\\
	\textbf{Special Requirements} & Der Student verfügt über einen Account der externen Plattform.\\
	\textbf{Frequency of Occurrence} & Rund jede Woche, wenn die Dozenten erneut neue Unterrichtsmaterialien bereitstellen. \\
	%\textbf{Open Issues} & FIXME \\
\end{tabulary}


\subsubsection{UC4: Synchronisation der Dateien}

\begin{tabulary}{\linewidth}{lp{30em}}
	\textbf{Goal} & Die Studentin synchronisiert alle benötigten Unterrichtsmaterialien auf ihren Computer. \\
	%\textbf{Level} & FIXME \\
	\textbf{Primary Actor} & Die Studentin \\
	\textbf{Trigger} & Die Studentin möchte aktuelle Unterrichtsmaterialien auf ihrem Computer lokal abspeichern. \\
	\textbf{Stakeholders and Interests} & \textbf{Die Studentin}: möchte stets die Version der Dateien auf ihrem Rechner haben, die sie tatsächlich will. \textbf{Das System:}  stellt sicher, dass die gespeicherten Dateien konsistent und aktuell sind.\\
	\textbf{Preconditions} & UC3: Die Studentin hat erfolgreich eine Verbindung zur externen Plattform hergestellt. \\
	%\textbf{Postconditions} & FIXME \\
	\textbf{Main Success Scenario} & Die Studentin startet den Synchronisationsprozess. Danach verfügt sie über die von ihr ausgewählten Unterrichtsmaterialien auf ihrem Rechner. Eine Log-Datei protokolliert den Vorgang. \\
	\textbf{Extensions} & \textbf{Dateien schon vorhanden:} Die Studentin legt in ihrem Profil fest, wie mit bereits existierenden Dateien umgegangen werden soll. Dabei kann sie wählen, ob lokal bestehende Dateien gleichen Namens umbenannt, durch eine neuere Server-Version ersetzt oder nichts synchronisiert werden soll. Sie kann auch entscheiden, bei jedem Konflikt von \emph{kitovu} gefragt zu werden, wie sie für jede Datei entscheiden möchte. \textbf{Synchronisation schlägt fehl:} Die Synchronisation wird nicht durchgeführt, die Studentin erhält eine Benachrichtigung.\\
	%\textbf{Special Requirements} & FIXME \\
	\textbf{Frequency of Occurrence} & Rund jede Woche, wenn die Dozenten erneut neue Unterrichtsmaterialien bereitstellen.  \\
	%\textbf{Open Issues} & Moodle-Anbindung \\
\end{tabulary}

\pagebreak
\section{Weitere Anforderungen}

\subsection{Qualitätsmerkmale}
%<Beschreibung der Qualitätsmerkmale der Software (Verweis auf ISO 9126 als Checkliste)>

Gemäss ISO 9126

\subsubsection{Funktionalität (functionality)}

\begin{description}
  \item[Angemessenheit (suitability)]
    FXME
  \item[Richtigkeit (accuracy)]
    FIXME
  \item[Interoperabilität (interoperability)]
    FIXME
  \item[Sicherheit (security)]
    FIXME
  \item[Ordnungsmässigkeit (functionality compliance)]
    FIXME
\end{description}

\subsubsection{Zuverlässigkeit (reliability)}

\begin{description}
  \item[Reife (maturity)]
    FXME
  \item[Fehlertoleranz (fault tolerance)]
    FIXME
  \item[Wiederherstellbarkeit (recoverability)]
    FIXME
  \item[Konformität (reliability compliance)]
    FIXME
\end{description}

\subsubsection{Benutzbarkeit (usability)}

\begin{description}
  \item[Verständlichkeit (understandability)]
    FIXME
  \item[Erlernbarkeit (learnability)]
    FIXME
  \item[Bedienbarkeit (operability)]
    FIXME
  \item[Attraktivität (attractiveness)]
    FIXME
  \item[Konformität (usability compliance)]
    FIXME
\end{description}

\subsubsection{Effizienz (efficiency)}

\begin{description}
  \item[Zeitverhalten (time behaviour)]
    FIXME
  \item[Verbrauchsverhalten (resource utilization)]
    FIXME
  \item[Konformität (efficiency compliance)]
    FIXME
\end{description}

\subsubsection{Änderbarkeit (maintainability)}

\begin{description}
  \item[Analysierbarkeit (analyzability)]
    FIXME
  \item[Modifizierbarkeit (changeability)]
    FIXME
  \item[Stabilität (stability)]
    FIXME
  \item[Testbarkeit (testability)]
    FIXME
  \item[Konfromität (maintainability compliance)]
    FIXME
\end{description}

\subsubsection{Übertragbarkeit (portability)}

\begin{description}
  \item[Anpassbarkeit (adaptability)]
    FIXME
  \item[Installierbarkeit (installability)]
    FIXME
  \item[Koexistenz (co-existence)]
    FIXME
  \item[Austauschbarkeit (replaceability)]
    FIXME
  \item[Konformität (Portability compliance)]
    FIXME
\end{description}

\subsection{Schnittstellen}
%<Beschreibung der Schnittstellen der Software>

Folgende Schnittstellen werden im Projekt \emph{kitovu} zu Verfügung gestellt oder verwendet:

\begin{description}
  \item[Benutzeroberfläche]
    Die Benutzeroberfläche ist die Schnittstelle, mit welcher die meisten Endbenutzer mit der Applikation interagieren wird.
    Diese bietet einen einfachen Weg, um die Daten zu synchronisieren.
  \item[Command-Line-Programm]
    Alternativ bieten wir ein Command-Line-Programm, welches ermöglicht über einfache Befehle die Synchronisation zu starten.
    Dies ist besonders für Automatisierung oder für Endbenutzer, welche die Kommandozeile bevorzugen, gedacht.
  \item[Externe Schnittstellen]
    Wir verwenden als externe Schnittstellen das Studentenportal, das Moodle und den Skripteserver, sowie das Switch AAI für die Authorisierung des Moodles.
\end{description}

\subsection{Randbedingungen}
%<Auflistung der wichtigsten Randbedingungen mit einer Beschreibung dazu>

Dies technischen Randbedingungen wurden für das Projekt \emph{kitovu} festgelegt:

\begin{description}
  \item[Python] Version 3.6
  \item[Betriebssystem] Hauptsächlich Linux, aber auch Windows und macOS unterstützt
\end{description}

\end{document}
